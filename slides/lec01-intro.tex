\PassOptionsToPackage{xetex}{xcolor}
\PassOptionsToPackage{xetex}{graphicx}
\documentclass[a4paper,landscape,headrule,footrule,xetex]{foils}

%%
%%% macros for 2009 Semester 1 HG 803
%%%
\newcommand{\logo}{~}
\newcommand{\header}[3]{%
  \title{\vspace*{-2ex} \large HG3051  Corpus Linquistics
    \\[2ex] \Large  \emp{#2} \\ \emp{#3}}
  \author{\blu{Francis Bond}   \\ 
    \normalsize  \textbf{Division of Linguistics and Multilingual Studies}\\
    \normalsize  \url{http://www3.ntu.edu.sg/home/fcbond/}\\
    \normalsize  \texttt{bond@ieee.org}}
  \MyLogo{HG3051 (2018)}
  \renewcommand{\logo}{#2}
  \hypersetup{
    pdfinfo={
      Author={Francis Bond},
      Title={#1: #2},
      Subject={HG3051: Corpus Linguistics},
      Keywords={Corpus Linguistics},
      License={CC BY 4.0}
    }
  }
  \date{#1 \\ \url{https://github.com/bond-lab/Corpus-Linguistics}}
}

\usepackage{fontenc}
\usepackage{polyglossia}
\setmainlanguage{english}
\setmainfont{TeX Gyre Pagella}
%\setmainfont{Linux Libertine}
%\setmainfont{Charis SIL}
\newfontfamily{\ipafont}{Gentium}
\newcommand{\ipa}[1]{{\ipafont\selectfont #1}}
\usepackage{xeCJK}

\setCJKmainfont{Noto Sans CJK SC}
\setCJKsansfont{Noto Sans CJK SC}



\usepackage{xcolor}
\usepackage{graphicx}
\newcommand{\blu}[1]{\textcolor{blue}{#1}}
\newcommand{\grn}[1]{\textcolor{green}{#1}}
\newcommand{\hide}[1]{\textcolor{white}{#1}}
\newcommand{\emp}[1]{\textcolor{red}{#1}}
\newcommand{\txx}[1]{\textbf{\textcolor{blue}{#1}}}
\newcommand{\lex}[1]{\textbf{\mtcitestyle{#1}}}

\usepackage{pifont}
\renewcommand{\labelitemi}{\textcolor{violet}{\ding{227}}}
\renewcommand{\labelitemii}{\textcolor{purple}{\ding{226}}}

\newcommand{\subhead}[1]{\noindent\textbf{#1}\\[5mm]}

\newcommand{\Bad}{\emp{\raisebox{0.15ex}{\ensuremath{\mathbf{\otimes}}}}}
\newcommand{\bad}{*}

\newcommand{\com}[1]{\hfill \textnormal{(\emp{#1})}}%
\newcommand{\cxm}[1]{\hfill \textnormal{(\txx{#1})}}%
\newcommand{\cmm}[1]{\hfill \textnormal{(#1)}}%

\usepackage{relsize,xspace}
\newcommand{\into}{\ensuremath{\rightarrow}\xspace}
\newcommand{\ent}{\ensuremath{\Rightarrow}\xspace}
\newcommand{\nent}{\ensuremath{\not\Rightarrow}\xspace}
\newcommand{\tot}{\ensuremath{\leftrightarrow}\xspace}
\usepackage{url}
\newcommand{\lurl}[1]{\MyLogo{\url{#1}}}

\usepackage{mygb4e}
\let\eachwordone=\itshape
\newcommand{\lx}[1]{\textbf{\textit{#1}}}

%\usepackage{times}
%\usepackage{nttfoilhead}
\newcommand{\myslide}[1]{\foilhead[-25mm]{\raisebox{12mm}[0mm]{\emp{#1}}}\MyLogo{\logo}}
\newcommand{\myslider}[1]{\rotatefoilhead[-25mm]{\raisebox{12mm}[0mm]{\emp{#1}}}}
%\newcommand{\myslider}[1]{\rotatefoilhead{\raisebox{-8mm}{\emp{#1}}}}

\newcommand{\section}[1]{\myslide{}{\begin{center}\Huge \emp{#1}\end{center}}}



\usepackage[lyons,j,e,k]{mtg2e}
\renewcommand{\mtcitestyle}[1]{\textcolor{teal}{\textsl{#1}}}
%\renewcommand{\mtcitestyle}[1]{\textsl{#1}}
\newcommand{\chn}{\mtciteform}
\newcommand{\cmn}{\mtciteform}
\newcommand{\iz}[1]{\textup{\texttt{\textcolor{blue}{\textbf{#1}}}}}
\newcommand{\rel}[1]{\textsc{\color{blue}{#1}}}
\newcommand{\wn}[3]{\lex{#1}\ensuremath{_{#2:#3}}}
\newcommand{\con}[1]{\textsc{#1}}
\newcommand{\gm}{\textsc}
\usepackage[normalem]{ulem}
\newcommand{\ul}{\uline}
\newcommand{\ull}{\uuline}
\newcommand{\wl}{\uwave}
\newcommand{\vs}{\ensuremath{\Leftrightarrow}~}
\usepackage[hidelinks]{hyperref}
\hypersetup{
     colorlinks,
     linkcolor={blue!50!black},
     citecolor={red!50!black},
     urlcolor={blue!80!black}
}
%%%
%%% Bibliography
%%%
\usepackage{natbib}
%\usepackage{url}
\usepackage{bibentry}
%%% From Tim
\newcommand{\WMngram}[1][]{$n$-gram#1\xspace}
\newcommand{\infers}{$\rightarrow$\xspace}


\header{Introduction to Corpus Linguistics}{Main Issues}


\begin{document}
\maketitle


\myslide{Introduction}

\begin{itemize}
%\item Self Introduction
\item Administrivia
\item What is Corpus Linguistics
\item What this course is (and isn't)
\item Getting to know each other (what do you want?)
\end{itemize}

% \myslide{Self Introduction}

% \begin{itemize}\addtolength{\itemsep}{-3mm}
% %\item Francis Bond 
% \item BA in Japanese and Mathematics 
% \item BEng in Power and Control %(Electrical Systems Engineering)
%   \raisebox{-2ex}[0mm][0mm]{\includegraphics{pwm.eps}}
% \item PhD on  ``Determiners and Number in English
%   contrasted with Japanese,  as exemplified in Machine
%   Translation'' 
% \item 1991-2006 NTT (Nippon Telegraph and Telephone)
%   \begin{itemize}
%   \item Japanese - English/Malay Machine Translation
%   \item Japanese corpus, grammar and ontology (Hinoki)
%   \end{itemize}
% \item 2006-2009 NICT (National Inst. for Info. and Comm. Technology)
%   \begin{itemize}
%   \item Japanese - English/Chinese Machine Translation
%   \item Japanese WordNet
%   \end{itemize}
% \item 2009- NTU
% \end{itemize}

\myslide{Corpora I have been involved with}

\begin{itemize}
\item Semantic markup of the LDC \textit{Call Home Corpus}
\\ sense tagging of Japanese telephone transcripts
\item \textit{Hinoki Treebank of Japanese}
\\ HPSG parses of Japanese definitions, examples and newspaper text
\\ sense tagging of same
\item \textit{Tanaka Corpus} of aligned Japanese-English text
\\ Now the \textit{Tatoeba multilingual project}: \url{www.tatoeba.org}
\item NICT English learner corpus (advisor)
\item \textit{Japanese WordNet gloss corpus}, jSEMCOR corpus
\\  aligned Japanese-English text, sense tagging
\end{itemize}

\myslide{Corpora I am building now}
\MyLogo{\citep{Tan:Bond:2012,NTUCLE:2017}}
\begin{itemize}
\item \textbf{\blu{NTU Multilingual Corpus: NTU-MC}}
\\ with help from Tan Liling and many students
  \begin{itemize}
  \item Arabic, Chinese, English, Indonesian, Japanese, Korean, Vietnamese
    \begin{itemize}
    \item Essays
    \item Short Stories (Sherlock Holmes, Karel Čapek)
    \item News Text
    \item Singapore Tourist Web Sites
    \end{itemize}
  \item Wordnet sense tagging, HPSG parses
  \item Cross lingual alignment
  \item Tagging various phenomena
  \item Used in URECA and FYP research, we will use it in this course
  \end{itemize}
\item \txx{NTU Corpus of Learner English: NTUCLE}
  \begin{itemize}
  \item With help from LCC
  \end{itemize}

\end{itemize}




\myslide{100\% Continuous Assessment}


\begin{itemize}
\item Individual Lab Work (4x10\%)
\item Individual Project (20\%) 
  \begin{itemize}
  \item  Describe some linguistic phenomenon quantitatively in a 6-page paper (ACL format)
  \item The paper must motivate both the choice of phenomenon and corpus
  \end{itemize}
\item Group Project (30\%) One of:
  \begin{itemize}
  \item A program to perform some substantial corpus processing task
  \item The collection and annotation of a new (sub)corpus
  \item[+]     8-page paper (ACL full paper format with extra page for references) describing your approach
  \end{itemize}
\item Class Participation  (10\%)
\end{itemize}


\myslide{Guidelines for Written Work}

\begin{itemize}
\item All assignments must follow the \textit{(Computational)
    Linguistic Style Guidelines: a guide for the perplexed. } \\
  \url{https://fcbond.github.io/static/img/ling-style.pdf}
\item Proper citation is important 
    \\ --- failure to cite is plagiarism --- \textbf{zero} or \textbf{fail}
\item Local Rules
  \begin{itemize}
  \item ACL format for paper submission (No need for separate title page)
    \\ only the first $n$ pages will be marked
  \item Late assignments get \textbf{zero}
  \item I expect some quantitative analysis
  \item I will try to give you real problems to work on
  \end{itemize}
\end{itemize}


\myslide{Extra Credit}

\begin{itemize}
\item If you submit a patch\footnote{a short set of commands to correct a bug in a computer program} that gets accepted to a corpus or tool we use
  \begin{itemize}
  \item you can get 1-5\% extra credit (depending on the size/difficulty)
    \\ typically 10$^{n-1}$ where $n$ is the number of lines you changed
  \item you can't go over 100\%
  \end{itemize}
\item A patch can involve
  \begin{itemize}
  \item extending the corpus/code with new capabilities
  \item fixing a bug in annotation/code
  \item fixing a bug in or extending documentation
    \begin{itemize}
    \item fixing a spelling error; rewording for clarity; translating to a new language
    \end{itemize}
  \end{itemize}
\item Has to be for this course (not overlap with another course or thesis, \ldots)
\end{itemize}


\myslide{The goal of this course}

\begin{center}
  \LARGE 
Master the uses of text corpora 
\\ in linguistics research and applications.
\end{center}
\begin{itemize}\addtolength{\itemsep}{-1ex}
\item Selecting text
\item Marking up extra information
\item The range of existing corpora
\item How to build your own corpus
\item Using corpora to test linguistic hypotheses
\item Using corpora to train language tools
\item Extracting knowledge from corpora
\end{itemize}


\myslide{Prerequisites}

\begin{itemize}
\item Some linguistic knowledge assumed
  \begin{itemize}
  \item You know what a \txx{lexeme} is
  \item You know what an \txx{inflectional paradigm} is
  \item You know what a \txx{constituent} is
  \end{itemize}
 If you don't know these, you will have to do a little background
 reading, I recommend \citet{Huddleston:1988} 
\item A little computational knowledge (not required but useful)
  \begin{itemize}
  \item You will learn some very simple techniques here
  \item You will learn to use some corpus programs
  \item If you can program a little \blu{I encourage you to use your skills}
  \end{itemize}
\end{itemize}

% \myslide{Course Content} 

% This course introduces basic corpus skills for linguists:
% \begin{itemize}
% \item Marking up extra information
% \item Selecting text
% \item The range of existing corpora
% \item How to build your own corpus
% \item Using corpora to test linguistic hypotheses
% \item Using corpora to train language tools
% \end{itemize}

\myslide{What do you learn?}

On completion of this module, students should be able to:

\begin{itemize}
\item Understand the uses of text corpora in language research
  \\ Be able to manipulate them with simple tools
\item Write simple SQL queries
\item Design and build a corpus for some task
\item Understand how to analyse corpus data through basic statistical methods
\end{itemize}
    
%%%
%%% this changes each year, so keep separate
%%%
\include{schedule}


\myslide{Textbook and Readings}

\begin{itemize}
\item I haven't found a perfect text book, so we won't use one.
  \begin{itemize}
  \item Stubbs, Michael, \textit{Text and Corpus Analysis}. Blackwell Publishers, 1996
\\ is not bad
  \end{itemize}
\item Readings will be assigned, I will try to choose
  works that are on-line.
\item All Wikipedia articles cited have been checked by me, and I will watch
  them for changes. \com{extend the web of trust}
\end{itemize}

% Other References

% Biber, D., S. Conrad \& R. Reppen, Corpus Linguistics: Investigating Language Structure and Use. Cambridge University Press, 1998.

% Kennedy, G. An Introduction to Corpus Linguistics. Longman, 1998.

% McEnery, Tony et al. Corpus-Based Language Studies: An Advanced Resource Book. Routledge, 2006.

% McEnery, Tony and Andrew Wilson Corpus Linguistics 2nd ed, Edinburgh UP, 2001

% Sinclair, John. Corpus Concordance Collocation. Oxford: Oxford UP, 1991


\myslide{Student Responsibilities}

By remaining in this class, the student agrees to:
\begin{enumerate}
\item  Make a good-faith effort to learn and enjoy the material.
\item  Read assigned texts and participate in class discussions and activities.
\item Submit assignments on time.
\item Attend class at all times, barring special circumstances (see below).
\item Get help early: approach me when you first have trouble understanding a concept or homework problem rather than complaining about a lack of understanding afterward.
\item Treat other students with respect in all class-related activities, including on-line discussions.
\end{enumerate}
\myslide{Attendance}
\begin{enumerate}
\item You are expected to attend all classes.
\item Be on time - lateness is disruptive to your own and others' learning.
\item Valid reasons for missing class include the following:
\begin{enumerate}
\item A medical emergency (including mental health emergencies)
\item A family emergency (death, birth, natural disaster, etc).
\end{enumerate}
\item There will be significant material covered in class that is not in your readings.  You cannot expect to do well without coming to class.
\item If you miss a class, it is your responsibility to get the notes, any handouts you missed, schedule changes, etc. from a classmate.
\end{enumerate}

\myslide{Remediation and Academic Integrity}
\begin{enumerate}
\item No late work will be accepted, except in the case of a documented excuse.
\item For planned, justified, absences on class days or days on which assignments are due, advance notice must be provided.
\item Cheating will not be tolerated. Violations, including plagiarism, will be seriously dealt with, and could result in \textbf{a failing grade for the entire course}.
\item For all other issues of academic integrity, refer to the University Honour Code
\item As always, use your common sense and conscience.
\end{enumerate}




\myslide{Why do \txx{you} do Corpus Linguistics?}

\begin{itemize}
\item Language Poll (What do you speak and/or study?)
  \begin{itemize}
  \item Natural
    \begin{itemize}
    \item Mandarin
    \item Bahasa Malay
    \item Tamil
    \item[\ldots]
    \end{itemize}
  \item Corpus Type
    \begin{itemize}
    \item Text
    \item Speech
    \item Other
    \item[\ldots]
    \end{itemize}
  \end{itemize}


\end{itemize}



\myslide{What is a Corpus?}

\lx{corpus} (pl: \lx{corpora}):
\begin{enumerate}
\item A collection of texts, especially if
complete and self-contained: the corpus of
Anglo-Saxon verse.
\item In linguistics and lexicography, a body of texts, utterances, or other
specimens considered more or less representative
of a language, and usually stored as an electronic
database. Currently, computer corpora may store
many millions of running words, whose features can
be analyzed by means of tagging (the addition of
identifying and classifying tags to words and other
formations) and the use of concordancing
programs. Corpus linguistics studies data in any
such corpus \ldots
\end{enumerate}
(from \textit{The Oxford Companion to the English Language}, ed.
McArthur \& McArthur, 1992)

\myslide{Definition of a corpus}
\begin{itemize}
\item  In principle, any collection of more than one text can be called a
\txx{corpus}
\item  Characteristics of modern corpora:
  \begin{itemize}
  \item machine-readable (i.e., computer-based)
  \item authentic (i.e., naturally occurring)
  \item sampled (bits of text taken from multiple sources)
  \item representative of a particular language or language variety.
  \end{itemize}
\item  \citet[171]{Sinclair:1991}:
  \begin{quote}
    A corpus is a collection of naturally-occurring language text, chosen to characterize a state of variety of language.
\end{quote}
% \item Corpus linguistics is the study of language as expressed in samples
% (corpora) or ``real world'' text.
\end{itemize}


\myslide{Why Are Electronic Corpora Useful?}
\begin{itemize}
\item as a collection of examples for linguists
\item as a data resource for lexicographers
\item as instruction material for language teachers and learners
\item as training material for natural language processing applications
  \begin{itemize}
  \item training of speech recognizers
  \item training of statistical part-of-speech taggers and parsers
  \item training of example-based and statistical machine translation systems
  \item training large language models
  \end{itemize}
\item ``Big Data'' is just another corpus, to analyze it wisely
  requires the same techniques

\end{itemize}



\myslide{Examples for Linguists}

Give examples for English noun phrases \ldots

\myslide{Examples for Linguists}
Examples from the Penn treebank:
\begin{exe}
  \ex \eng{USX 's transition from Big Steel to Big Oil}
  \ex \eng{Pittsburgh instead of New York or Findlay,  Ohio, Marathon 's home}
  \ex \eng{his concern about boosting shareholder value}
  \ex \eng{the modest goal of becoming tax manager by the age of 46}
  \ex \eng{a move that, in effect, raised the cost of a \$7.19 billion Icahn bid by about \$3
  billion}
\ex \eng{an undistinguished college student who dabbled in zoology until he concluded that he couldn't stand cutting up frogs }
\ex \eng{the sale of the reserves of Texas Oil \& Gas, which was acquired three years ago and
hasn't posted any significant operating profits since}
\end{exe}



\myslide{Some Linguists dismiss Corpus Linguistics}

\begin{quotation}
\ldots it is obvious that the set of grammatical
sentences cannot be identified with any particular corpus
of utterances \ldots  

\bigskip
\ldots a grammar mirrors the behavior of the speaker, who,
on the basis of a finite and accidental experience with
language, can produce or understand an indefinite
number of new sentences.   

\bigskip
\ldots  ones's ability to produce and recognize
grammatical utterances is not based on notions of
statistical approximations or the like.
\ldots  If we rank the sequences of a given length in order of
statistical approximation to English, we will find both
grammatical and ungrammatical sequences scattered
throughout the list; there appears to be no particular
relation between the order of approximations and
grammatical.  
\\ \mbox{} \hfill \citet[pp15--17]{Chomsky:1957} \textit{Syntactic Structures}  
\end{quotation}

\myslide{Can grammaticality be predicted?}

%\myslide{What is grammatical and what isn't?}
 \begin{exe}
 \ex\label{gram} {~}\eng{Colorless green ideas sleep furiously.}
 \ex\label{ungram} *\eng{Furiously sleep ideas green colorless.}
 \hfill (Chomsky, 1957: as (1) and (2))
 \end{exe}
 \begin{quote}
   It is fair to assume that neither sentence (\ref{gram}) nor
   (\ref{ungram}) (nor indeed any part of these sentences) has ever
   occurred in an English discourse. Hence, in any statistical model
   for grammaticalness, these sentences will be ruled out on identical
   grounds as equally `remote' from English. Yet (\ref{gram}), though
   nonsensical, is grammatical, while (\ref{ungram}) is not.
 \end{quote}

\txx{Not really:}
Using a simple probabilistic model (based only on the probability of a
word occurring given the two proceeding words) \citet{Pereira:2000} showed
that P(\ref{gram}) $\gg$ P(\ref{ungram}) ($\times 200,000$).

\myslide{Context helps}
\MyLogo{\url{http://www.linguistlist.org/issues/2/2-457.html}}
\begin{quotation}
  \eng{It can only be the thought of verdure to come, which prompts us in
  the autumn to buy these dormant white lumps of vegetable matter
  covered by a brown papery skin, and lovingly to plant them and care
  for them. It is a marvel to me that under this cover they are
  labouring unseen at such a rate within to give us the sudden awesome
  beauty of spring flowering bulbs. While winter reigns the earth
  reposes but these colourless green ideas sleep furiously.}
  \emph{C.M Street (1985)}
\end{quotation}

\myslide{Why do Linguists need Corpora?}

\begin{description}
\item[Chomksy] The verb \lex{perform} cannot be used with mass word objects: one can \eng{perform a task} but not \eng{perform labour}.
\item[Hatcher] How do you know, if you don't use a corpus and have not studied the verb \lex{perform}?
\item[Chomksy] How do I know?  Because I am a native speaker of the English Language.
\end{description}
Hill (1962:29) cited in \citet[11]{McEnery:Wilson:2001}

\myslide{This is why}% English C\myslide{orpora}
From the BNC (search for ``\texttt{perform [nn1*]}'')
\begin{verbatim}
PERFORM MUSIC 	4
PERFORM WORK 	4
PERFORM SURGERY 	3
PERFORM EUTHANASIA 	2
PERFORM RESEARCH        2
\end{verbatim}

\eng{many Continental musicians, and it can not be doubted that professional English singers often \ul{perform music} which they have not had time to " learn " in any sense of}

\eng{Not only do `` Saxtet '' \ul{perform music} previously unassociated with the saxophone, but they include a selection of their own}

Linguists' intuitions are unreliable: \blu{Explanations of languages
  based on false data are not very valuable.}


\myslide{Examples for Lexicographers}

How many senses does the word \lex{line} have?

\myslide{Examples for Lexicographers}

The noun line has 30 senses according to WordNet (first 23 from tagged texts):

\begin{enumerate}
\item (51) \lex{line} --- (a formation of people or things one beside another; \eng{the \ul{line} of soldiers advanced with their bayonets fixed}; \eng{they were arrayed in \ul{line} of battle}; \eng{the cast stood in \ul{line} for the curtain call})
\item (20) \lex{line} --- (a mark that is long relative to its width; \eng{He drew a \ul{line} on the chart})
\item (15) \lex{line} --- (a formation of people or things one behind another; \eng{the \ul{line} stretched clear around the corner}; \eng{you must wait in a long \ul{line} at the checkout counter})
\item (13) \lex{line} --- (a length (straight or curved) without breadth or thickness; the trace of a moving point)
\item (11) \lex{line} --- (text consisting of a row of words written across a page or computer screen; \eng{the letter consisted of three short \ul{line}s}; \eng{there are six \ul{line}s in every stanza})
\item (10) \lex{line} --- (a single frequency (or very narrow band) of radiation in a spectrum)
\item (10) \lex{line} --- (a fortified position (especially one marking the most forward position of troops); \eng{they attacked the enemy's \ul{line}})
\item (10) argumentation, logical argument, argument, \ul{line} of reasoning, \lex{line} --- (a course of reasoning aimed at demonstrating a truth or falsehood; the methodical process of logical reasoning; \eng{I can't follow your \ul{line} of reasoning})
\item (9) cable, \lex{line}, transmission line --- (a conductor for transmitting electrical or optical signals or electric power)
\item (8) course, \lex{line} --- (a connected series of events or actions or developments; \eng{the government took a firm course}; \eng{historians can only point out those \ul{line}s for which evidence is available})
\item (6) \lex{line} --- (a spatial location defined by a real or imaginary unidimensional extent)
\item (5) wrinkle, furrow, crease, crinkle, seam, \lex{line} --- (a slight depression in the smoothness of a surface; \eng{his face has many \ul{line}s}; \eng{ironing gets rid of most wrinkles})
\item (4) pipeline, \lex{line} --- (a pipe used to transport liquids or gases; \eng{a pipeline runs from the wells to the seaport})
\item (4) \lex{line}, railway line, rail line --- (the road consisting of railroad track and roadbed)
\item (3) telephone line, phone line, telephone circuit, subscriber line, \lex{line} --- (a telephone connection)
\item (3) \lex{line} --- (acting in conformity; \eng{in line with}; \eng{he got out of \ul{line}}; \eng{toe the \ul{line}})
\item (2) lineage, \lex{line}, line of descent, descent, bloodline, blood line, blood, pedigree, ancestry, origin, parentage, stemma, stock -- (the descendants of one individual; \eng{his entire lineage has been warriors})
\item (2) \lex{line} --- (something (as a cord or rope) that is long and thin and flexible; \eng{a washing \ul{line}})
\item (2) occupation, business, job, line of work, \lex{line} --- (the principal activity in your life that you do to earn money; \eng{he's not in my \ul{line} of business})
\item (1) \lex{line} --- (in games or sports; a mark indicating positions or bounds of the playing area)
\item (1) channel, communication channel, \lex{line} --- ((often plural) a means of communication or access; \eng{it must go through official channels}; \eng{lines of communication were set up between the two firms})
\item (1) \lex{line}, product line, line of products, line of merchandise, business line, line of business -- (a particular kind of product or merchandise; \eng{a nice \ul{line} of shoes})
\item (1) \lex{line} --- (a commercial organization serving as a common carrier)
\item agate line, \lex{line} --- (space for one line of print (one column wide and 1/14 inch deep) used to measure advertising)
\item credit line, line of credit, bank line, \lex{line}, personal credit line, personal line of credit -- (the maximum credit that a customer is allowed)
\item tune, melody, air, strain, melodic line, \lex{line}, melodic phrase -- (a succession of notes forming a distinctive sequence; \eng{she was humming an air from Beethoven})
\item \lex{line} --- (persuasive but insincere talk that is usually intended to deceive or impress; \eng{`let me show you my etchings' is a rather worn \ul{line}}; \eng{he has a smooth \ul{line} but I didn't fall for it}; \eng{that salesman must have practiced his fast \ul{line} of talk})
\item note, short letter, \lex{line}, billet -- (a short personal letter; \eng{drop me a \ul{line} when you get there})
\item \lex{line}, dividing line, demarcation, contrast -- (a conceptual separation or distinction; \eng{there is a narrow \ul{line} between sanity and insanity})
\item production line, assembly line, \lex{line} --- (mechanical system in a factory whereby an article is conveyed through sites at which successive operations are performed on it)
\end{enumerate}

\myslide{Instruction for Language Learning}

Which do you say in English: \eng{think about} or \eng{think on}?
\myslide{Instruction for Language Learning}

Which do you say in English: \eng{think about} or \eng{think on}?

If in doubt, ask google:
\begin{tabular}[t]{rl}
 36,300,000 hits & \eng{think about} \\
 738,000 hits &  \eng{think on}  
\end{tabular}


\myslide{Types of Corpora}
\begin{itemize}
\item mono-lingual versus multi-lingual corpora
\item special-purpose, domain-specific corpora versus general-purpose, large-scale corpora
\item spoken language corpora versus collections of written text
\item ad-hoc corpus collections versus balanced, representative corpora
\item raw text versus marked-up documents
\item unannotated versus annotated corpora
\item Web as a corpus
\end{itemize}

\myslide{What does a corpus consist of?}

\begin{itemize}
\item A collection of ordinary text files (Raw Corpus)
\item Annotated corpora
  \begin{itemize}
  \item Raw corpora with html/xml tags (genre, date, subject, \ldots)
  \item Annotated corpora (part of speech, syntactic structures, etc.)
  \end{itemize}
\end{itemize}  

\myslide{The British National Corpus (BNC)}
\begin{itemize}
\item 100 million words of written and spoken British English \citep{Burnard:2000}
\item Designed to represent a wide cross-section of British English from late 20th century: balanced and representative
\item POS tagging (2 million word sampler hand checked)
\end{itemize}
\begin{small}

  \begin{tabular}{l|lll}
    Written  & Domain  & Date & Medium \\ \hline
    (90\%)  & Imaginative (22\%) & 1960-74 (2\%) & Book (59\%) \\
    & Arts (8\%) & 1975-93 (89\%)  & Periodical (31\%)   \\
    & Social science (15\%)  & Unclassified (8\%)  & Misc. published (4\%) \\
    & Natural science (4\%) \ldots  & & Misc. un-pub (4\%)    \\  \hline
    Spoken  & Region  & Interaction type  & Context-governed \\ \hline
    (10\%)  &  South (46\%)  & Monologue (19\%)  & Informative (21\%) \\
    & Midlands (23\%)  & Dialogue (75\%)  & Business (21\%) \\
    & North (25\%)  \ldots & Unclassified (6\%)  & Institutional (22\%)  \ldots\\
  \end{tabular}
\end{small}


\myslide{General vs. specialized corpora}
\begin{itemize}
\item General corpora (such as “national” corpora) are a huge undertaking.
These are built on an institutional scale over the course of many years.
\item  Specialized corpora (ex: corpus of English essays written by Japanese
university students, medical dialogue corpus) can be built relatively
quickly for the purpose at hand, and therefore are more common
\item Characteristics of corpora:
  \begin{enumerate}
  \item Machine-readable, authentic
  \item Sampled to be balanced and representative
  \end{enumerate}
\newpage
\item  Trend: for specialized corpora, criteria in (2) are often weakened in favor of quick assembly and large size
  \begin{itemize}
  \item Do-it-yourself corpora
  \item World-Wide Web as a corpus
  \item Google 1T corpus
  \end{itemize}
 Rare phenomena only show up in large collections
\end{itemize}

\myslide{A short list of well-known corpora}
\begin{itemize}
\item \blu{National corpora}:
  \begin{itemize}
  \item The British National Corpus
  \item The American National Corpus
  \item The Czech National Corpus
  \item King Sejong the Great Corpus
  \end{itemize}
  Chinese, Greek, Italian, Hungarian, Polish, Czech \ldots
\item \blu{Other Well Known Corpora}:
  \begin{itemize}
  \item Brown Corpus
  \item Corpus of Contemporary American English
  \item  Michigan Corpus of Academic Spoken English
\newpage
  \item  1st Language acquisition:
    \begin{itemize}
    \item  CHILDES (Child Language Data Exchange System)
    \end{itemize}
  \item  2nd Language acquisition (mostly English)
    \begin{itemize}
    \item  ICLE (the International Corpus of Learner English) and LOCNESS (the Louvain Corpus of Native English Essays)
    \item  Longman Learners' Corpus
    \item  CLC (Cambridge Learner Corpus)
    \end{itemize}
  \item Multilingual Corpora
    \begin{itemize}
    \item Canadian Hansard
    \item Hong Kong Hansard
    \item Europarl
    \end{itemize}
  \item Parsed  Corpora
    \begin{itemize}
    \item Penn Treebank (WSJ, Brown, Chinese)
    \item Czech Dependency Bank
    \item Redwoods HPSG corpus of English
    \end{itemize}
\end{itemize}
\end{itemize}

\myslide{See Also}

\begin{itemize}
\item Linguist list corpora page
\\ \url{https://www.linguistlist.org/sp/GetWRListings.cfm?wrtypeid=1}
\item ACL Siglex Links to the CORPORA Mailing List Archive
\item Linguistics Data Consortium (LDC)
\item European Language Resources Association (ELRA)
\item \ja{言語資源協会} \textit{Gengo Shigen Kyouyuukikou} Language Resource Consortium (GSK)
\item \zh{中文语言资源联盟} Chinese Linguistic Data Consortium (CLDC)
\end{itemize}

% \myslide{Corpora at NTU}

% \begin{itemize}
% %\item Collins WordBank (NTU Library$|$Databases)
% %\item BNC (online)
% \item Cantonese Corpus (KK)
% \\ Now at \href{https://github.com/fcbond/hkcancor}{https://github.com/fcbond/hkcancor}
% \item Tatoeba Japanese-English (FCB)
% \item Various small corpora (AC, FK)
% \item NTU Multilingual Corpus (under construction: FCB)
% \item NTU Learner Corpus (under construction: FCB)
% \item \emp{We may add to these in this class}
% \end{itemize}

\myslide{Let's Explore}
\MyLogo{Inspired by \url{davies-linguistics.byu.edu/ling485/projects/p01.htm}}

Go to the Brigham-Young University interface to the BNC (see web page for logon details).


\begin{description}
\item [MORPHOLOGY]: Look for words starting with the prefix \eng{dis-}
  (e.g. \eng{dissent}).  What are the three most common singular nouns
  (\url{dis*.[nn1]}), the three most common adjectives (\url{dis*.[j*]}), and the
  three most common infinitival verbs (\url{dis*.[vvi]})

\item [LEXICAL]: Search for \eng{robot} [using CHARTS] and then
  compare the frequency in the five main genres.  In which genre is it
  the most/least common? In which sub-genre is it the most common
  (click on [SEE ALL SECTIONS]
\newpage

 \item [COLLOCATIONS]: What are the 5 most frequent adjectives with
   \eng{curry} as a noun (\url{curry.[nn*]})? (CONTEXT = [j*], [4] [4], [SORT] =
   [FREQUENCY]). Now change to [SORT] = [RELEVANCE]. What are the five
   most highly-ranked adjectives. What has changed, and why?

\item[GRAMMATICAL]: In which genre is the present perfect (\url{has
    [vvn*]}) and the past perfect (\url{had [vvn*]}) most common? Any
  idea why? (you can use CHART)

\item[LEXICO-GRAMMAR]: Look at the top five adjectives following
  \eng{come} and \eng{go} ( use [COMPARE WORDS]; WORD(S) = come , go;
  CONTEXT =  [j*] [0] [2]). Is there any pattern in terms of which
  adjectives occur with the two verbs?

\newpage
\item[SEMANTICS]: Compare the collocates of \eng{find} and
  \eng{discover} ( use [COMPARE WORDS]; WORD(S) = find.[v*] ,
  discover.[v*]; CONTEXT = [nn*] [0] [2]). Any patterns here?

\item[LEXICO-GRAMMAR]: Compare the five most common phrases with
  \url{we [v*]} in SPOKEN vs ACADEMIC.  What is the major difference
  between the two registers? (use LIST with SECTIONS)

\item[LEXICAL]: Compare the most frequent singular and plural nouns
  (\url{[nn1*]} and \url{[nn2*]}) in MAGAZINE vs ACADEMIC).  Which
  types are more common in each register? (Use two searches in LIST)
\end{description}




\myslide{Acknowledgments}


\begin{itemize}
\item Thanks to Na-Rae Han for 
  inspiration for some of the slides (from  \textit{LING 2050 Special Topics in Linguistics: Corpus linguistics}, U Penn) and also for the Student Policies (adapted).
\item Thanks to Sandra K\"{u}bler for some of the slides from her 
\textit{RoCoLi\footnote{Romania Computational Linguistics Summer School} Course: Computational Tools for Corpus Linguistics}
\item Thanks to Mark Davies (BYU) for the exploration ideas.
\item Definitions from WordNet 3.0
\end{itemize}


\small
\bibliographystyle{aclnat}
\bibliography{abb,mtg,nlp,ling}

\end{document}

 
%%% Local Variables: 
%%% coding: utf-8
%%% mode: latex
%%% TeX-PDF-mode: t
%%% TeX-engine: xetex
%%% LaTeX-section-list:  (("myslide" 1))
%%% End: 