\PassOptionsToPackage{xetex}{xcolor}
\PassOptionsToPackage{xetex}{graphicx}
\documentclass[a4paper,landscape,headrule,footrule,xetex]{foils}


%%
%%% macros for 2009 Semester 1 HG 803
%%%
\newcommand{\logo}{~}
\newcommand{\header}[3]{%
  \title{\vspace*{-2ex} \large HG3051  Corpus Linquistics
    \\[2ex] \Large  \emp{#2} \\ \emp{#3}}
  \author{\blu{Francis Bond}   \\ 
    \normalsize  \textbf{Division of Linguistics and Multilingual Studies}\\
    \normalsize  \url{http://www3.ntu.edu.sg/home/fcbond/}\\
    \normalsize  \texttt{bond@ieee.org}}
  \MyLogo{HG3051 (2018)}
  \renewcommand{\logo}{#2}
  \hypersetup{
    pdfinfo={
      Author={Francis Bond},
      Title={#1: #2},
      Subject={HG3051: Corpus Linguistics},
      Keywords={Corpus Linguistics},
      License={CC BY 4.0}
    }
  }
  \date{#1 \\ \url{https://github.com/bond-lab/Corpus-Linguistics}}
}

\usepackage{fontenc}
\usepackage{polyglossia}
\setmainlanguage{english}
\setmainfont{TeX Gyre Pagella}
%\setmainfont{Linux Libertine}
%\setmainfont{Charis SIL}
\newfontfamily{\ipafont}{Gentium}
\newcommand{\ipa}[1]{{\ipafont\selectfont #1}}
\usepackage{xeCJK}

\setCJKmainfont{Noto Sans CJK SC}
\setCJKsansfont{Noto Sans CJK SC}



\usepackage{xcolor}
\usepackage{graphicx}
\newcommand{\blu}[1]{\textcolor{blue}{#1}}
\newcommand{\grn}[1]{\textcolor{green}{#1}}
\newcommand{\hide}[1]{\textcolor{white}{#1}}
\newcommand{\emp}[1]{\textcolor{red}{#1}}
\newcommand{\txx}[1]{\textbf{\textcolor{blue}{#1}}}
\newcommand{\lex}[1]{\textbf{\mtcitestyle{#1}}}

\usepackage{pifont}
\renewcommand{\labelitemi}{\textcolor{violet}{\ding{227}}}
\renewcommand{\labelitemii}{\textcolor{purple}{\ding{226}}}

\newcommand{\subhead}[1]{\noindent\textbf{#1}\\[5mm]}

\newcommand{\Bad}{\emp{\raisebox{0.15ex}{\ensuremath{\mathbf{\otimes}}}}}
\newcommand{\bad}{*}

\newcommand{\com}[1]{\hfill \textnormal{(\emp{#1})}}%
\newcommand{\cxm}[1]{\hfill \textnormal{(\txx{#1})}}%
\newcommand{\cmm}[1]{\hfill \textnormal{(#1)}}%

\usepackage{relsize,xspace}
\newcommand{\into}{\ensuremath{\rightarrow}\xspace}
\newcommand{\ent}{\ensuremath{\Rightarrow}\xspace}
\newcommand{\nent}{\ensuremath{\not\Rightarrow}\xspace}
\newcommand{\tot}{\ensuremath{\leftrightarrow}\xspace}
\usepackage{url}
\newcommand{\lurl}[1]{\MyLogo{\url{#1}}}

\usepackage{mygb4e}
\let\eachwordone=\itshape
\newcommand{\lx}[1]{\textbf{\textit{#1}}}

%\usepackage{times}
%\usepackage{nttfoilhead}
\newcommand{\myslide}[1]{\foilhead[-25mm]{\raisebox{12mm}[0mm]{\emp{#1}}}\MyLogo{\logo}}
\newcommand{\myslider}[1]{\rotatefoilhead[-25mm]{\raisebox{12mm}[0mm]{\emp{#1}}}}
%\newcommand{\myslider}[1]{\rotatefoilhead{\raisebox{-8mm}{\emp{#1}}}}

\newcommand{\section}[1]{\myslide{}{\begin{center}\Huge \emp{#1}\end{center}}}



\usepackage[lyons,j,e,k]{mtg2e}
\renewcommand{\mtcitestyle}[1]{\textcolor{teal}{\textsl{#1}}}
%\renewcommand{\mtcitestyle}[1]{\textsl{#1}}
\newcommand{\chn}{\mtciteform}
\newcommand{\cmn}{\mtciteform}
\newcommand{\iz}[1]{\textup{\texttt{\textcolor{blue}{\textbf{#1}}}}}
\newcommand{\rel}[1]{\textsc{\color{blue}{#1}}}
\newcommand{\wn}[3]{\lex{#1}\ensuremath{_{#2:#3}}}
\newcommand{\con}[1]{\textsc{#1}}
\newcommand{\gm}{\textsc}
\usepackage[normalem]{ulem}
\newcommand{\ul}{\uline}
\newcommand{\ull}{\uuline}
\newcommand{\wl}{\uwave}
\newcommand{\vs}{\ensuremath{\Leftrightarrow}~}
\usepackage[hidelinks]{hyperref}
\hypersetup{
     colorlinks,
     linkcolor={blue!50!black},
     citecolor={red!50!black},
     urlcolor={blue!80!black}
}
%%%
%%% Bibliography
%%%
\usepackage{natbib}
%\usepackage{url}
\usepackage{bibentry}
%%% From Tim
\newcommand{\WMngram}[1][]{$n$-gram#1\xspace}
\newcommand{\infers}{$\rightarrow$\xspace}

\usepackage{bibentry}
\renewcommand{\cite}{\bibentry}

\header{Lecture 4}{Survey of Available Corpora}{}

\usepackage{pst-node}
\newcommand{\sa}[2]{\rnode{c#1}{\iz{#2}}}%\nodebox{c#1}}

%\usepackage{hieroglf}
\usepackage{wasysym}
%\newcommand{\grn}[1]{\textcolor{PineGreen}{#1}}
\newcommand{\ont}[1]{\textcolor{blue}{#1}}
\newcommand{\jcy}[1]{\textcolor{orange}{#1}}
\newcommand{\lxd}[1]{\textcolor{brown}{#1}}

\newcommand{\hinoki}{\grn{Hinoki}\xspace}
\newcommand{\lexeed}{\lxd{Lexeed}\xspace}
\newcommand{\jacy}{\jcy{JACY}\xspace}
\newcommand{\onto}{\ont{Ontology}\xspace}
%\newcommand{\itsdb}{\textsf{[incr tsdb()]}\xspace}
\newcommand{\GT}{Goi-Taikei\xspace}


\begin{document}

\bibliographystyle{apalike}
\nobibliography{abb,mtg,nlp,ling}
\maketitle


\myslide{Overview}

\begin{itemize} 
\item Revision 
  \begin{itemize}
  \item \blu{Multi-modal Corpora}
  \item \blu{Multi-lingual Corpora}
  \end{itemize}
\item Survey of Corpora
  \begin{itemize}
  \item Corpora that caught your interest
  \item Corpora that caught my interest
  \end{itemize}
\end{itemize}

%%%
%%% this changes each year, so keep separate
%%%
\include{schedule}

%%
\section{Revision of \\
Multi-modal  and Multi-lingual Corpora}
%%%


\myslide{Multi-modal Corpora}

\begin{itemize}
\item Language is not the only channel for communication: It is often
  combined with other modalities
  \begin{itemize}
  \item speech
  \item gesture
  \item facial expression
  \item gaze
  \item body posture
\\ ECG (Electrocardiogram), HR (Heart Rate), GSR (Galvanic Skin Response)
  \item activity: nursing, drawing, building
  \end{itemize}
\item Corpora that include more than one of these are \blu{multi-modal}
\end{itemize}

\myslide{HCRC Map Task}
\MyLogo{\url{http://groups.inf.ed.ac.uk/maptask/}}
\begin{itemize}
\item Early, influential dialog corpus (with maps)
  \begin{itemize}
  \item Two speakers sit opposite one another
  \item Each has a (different) map which the other cannot see
  \item One explains the route to the other
  \end{itemize}
\item Conditions
  \begin{itemize}
  \item familiar (friends) vs non-familiar
  \item gaze vs no-gaze
  \end{itemize}
\item Landmarks chosen for phonetic properties
\item Annotation: POS, Parse, Discourse structure, Gaze
\item Now replicated in many languages and dialects
\end{itemize}

\myslide{Other Multimodal Corpora}
\MyLogo{}
\begin{itemize}
\item E-Nightingale: Nursing Task Corpus
\begin{itemize}
\item Japanese project to analyze Nursing tasks and dialogs
\item recorder worn all day
\item beeps at ten minute intervals  (event-driven recording)
\end{itemize}
\item Many Meeting Corpora
  \begin{itemize}
  \item VACE Multimodal Meeting Corpus
    \begin{itemize}
    \item extra linguistic information very important
    \end{itemize}
  \end{itemize}
\end{itemize}  
\myslide{Multi-Lingual Corpora}

\begin{itemize}
\item Multilingual corpora are useful for
  \begin{itemize}
  \item Contrastive linguistic analysis
  \item Language learning
  \item Machine translation training
  \end{itemize}
\item Well known Corpora
  \begin{itemize}
  \item Europarl: 20+ languages;  18-40 million words, .6--1.3 million sentences
  \item OPUS:  On-line collection of multilingual text
  \item Taoteba:  User generated corpus of example sentences
  \item Canadian Hansard
  \item Hong Kong Hansard
  \item Bible Translation Corpus
  \item GALE Chinese-English, Japanese-English (DoD)
  \item NICT Japanese-English, Japanese-Chinese
  \end{itemize}
\end{itemize}

\myslide{Multilingual Corpus Construction}
\begin{itemize}
\item Other languages used as annotation
  \begin{itemize}
  \item Assumed high quality
  \end{itemize}
\item Other construction often done automatically
\item Article, Sentence and Word alignment
  \begin{itemize}
  \item Length/Structure based cues
  \item Lexically based cues
  \end{itemize}
\item Driven by MT research
\item Not so much high quality alignable multi-lingual text
\end{itemize}

\section{Your Corpora}

\myslide{Slides}
\MyLogo{When you are really famous, it's OK}
\begin{itemize}\addtolength{\itemsep}{-1.5ex}
\item \blu{You need to learn to follow instructions} ($\frac{-1}{10}$/issue)
  \begin{itemize}
  \item pdf not powerpoint
  \item no more than 5 pages
  \item Name on page one
  \item Deadline is not negotiable --- late revisions don't count 
  \end{itemize}
\item For paper/grant submissions 
\\ --- your submission paper will be rejected without review
\item In the workplace
\\ --- you will get shouted at and made to do it again
\item Why is it so important?
  \begin{itemize}
  \item The reviewer/boss has to read/process many, many submissions
  \item Anything that distracts them/takes extra time is bad
  \end{itemize}
\item Even if you forget everything about Corpora, \blu{remember this lesson}
\end{itemize}

\myslide{Presentations}
\MyLogo{}
\begin{itemize}
\item 5 minutes + 2 minutes for question
  \begin{itemize}
  \item I will indicate time at 4:00 and 5:00 (and 7:00)
  \item You \emp{must} stop talking at 5:00
  \end{itemize}
\item Everyone must ask one (new) question!
\item We need to keep time strictly
  \begin{itemize}
  \item Choose one or two points about your corpus to emphasize
  \end{itemize}
\end{itemize}

\section{My Corpora}

\myslide{Some corpora of interest (and why)}

\begin{itemize}
\item Open American National Corpus
\item Corpus of Hong Kong Cantonese 香港粵語語料庫 (by KK Luke)
\item Hinoki Treebank of Japanese
\item Redwoods Treebank of English
\item Tatoeba Corpus
\item NICT Multilingual Corpus/Kyoto Corpus
\item NTU Multilingual Corpus
\end{itemize}

\myslide{Open American National Corpus}
\lurl{http://www.anc.org/OANC/}

\begin{tabular}{llrr}
Name & Domain & No. files & No. words \\
\hline
charlotte & 	face to face 	& 93 &198,295 \\
switchboard & 	telephone	& 2,307 &3,019,477 \\
\hline
911  report & 	government, technical 	& 17  &	 281,093 \\
berlitz & 	travel guides 	& 179 &1,012,496 \\
biomed & 	technical	& 837 &3,349,714 \\
eggan & 	fiction 	& 1 &61,746 \\
icic & 	letters	& 245 &91,318 \\
oup & 	non-fiction	& 45 &330,524 \\
plos & 	technical	& 252 &409,280 \\
slate & 	journal	& 4,531 &4,238,808 \\
verbatim & 	journal	& 32 &582,384 \\
web  data &	government	& 285 &1,048,792 \\
\hline
Total& & 8,832 & 	14,623,927\\
\end{tabular}

A large collection of freely available data

\myslide{Creation/Annotation}

\begin{itemize}\addtolength{\itemsep}{-1ex}
\item OANC Annotation
  \begin{itemize}
  \item  Structural markup (sections, chapters, \ldots, paragraph, sentence)
  \item  Words (tokens) with part of speech annotations using the Penn tagset
  \item  Noun, Verb chunks
\end{itemize}
\item Contributed annotations
  \begin{itemize}
  \item BBN Named Entities (inline format)
  \item Syntactic parses
    \begin{itemize}
    \item Charniak constituency-based parser (Charniak \& Johnson, 2005)
    \item LTH dependency converter (Johansson \& Nugues, 2007)
    \item MaltParser (Nivre et al., 2007).
    \item English Resource Grammar \citep{Flickinger:2008}
    \end{itemize}
  \item Slate coreference (anaphora) annotations
  \item CLAWS part of speech tags
  \end{itemize}
\end{itemize}

\myslide{Manually Annotated Sub-Corpus (MASC)}

\begin{itemize}
\item 82,000 words drawn from the OANC
  \begin{itemize}
  \item WordNet senses
  \item FrameNet frame annotations
  \item Validated annotations for token and sentence boundaries, part of speech, noun chunks, verb chunks, named entities, and Penn Treebank syntactic annotation
  \item Language Understanding Corpus annotations
  \item Opinion, PropBank, and TimeML are either included in MASC I or forthcoming
  \end{itemize}
\item  All annotations are in LAF/GrAF format and can therefore be merged or combined using the ANC Tool and transduced to other formats using ANC2Go. 
\end{itemize}
  
\myslide{References}
\begin{itemize} \small
\item $n$-gram search \url{http://www.americannationalcorpus.org/OANC/ngram.html}
\item  Ide, N., Baker, C., Fellbaum, C., Passonneau, R. (2010). The Manually Annotated Sub-Corpus: A Community Resource For and By the People. Proceedings of the 48th Annual Meeting of the Association for Computational Linguistics, Uppsala, Sweden.
\item  Ide, N., Suderman, K., Simms, B. (2010). ANC2Go: A Web Application for Customized Corpus Creation. Proceedings of the Seventh Language Resources and Evaluation Conference (LREC 2010), Valletta, Malta.
\item  Ide, N. (2008). The American National Corpus: Then, Now, and Tomorrow. In Michael Haugh, Kate Burridge, Jean Mulder and Pam Peters (eds.), Selected Proceedings of the 2008 HCSNet Workshop on Designing the Australian National Corpus: Mustering Languages, Cascadilla Proceedings Project, Sommerville, MA. 
\end{itemize}


\myslide{Corpus of Hong Kong Cantonese 香港粵語語料庫}
\lurl{http://www.hku.hk/hkcancor/}
\begin{itemize}
\item 180,000-word corpus
\item 52 spontaneous conversations
\item 42 radio programmes
\item Segmented; POS tagged; Romanized
\item Available directly for download (no explicit license)
\item Produced by KK Luke
\end{itemize}
\myslide{Creation/Annotation}
\begin{itemize}
\item 30 hours of recordings (March 1997 --- August 1998)
\item Native speakers of Cantonese
\item ordinary settings with family members, friends and colleagues talking with each other freely on everyday topics such as current affairs, work and study, and personal hobbies
\item Some parts selected
\end{itemize}
\myslide{Usage}
\begin{itemize}
\item Used to examine the uses of the frequently used sentence final particles wo3 and bo3 in the 1990s in Hong Kong Cantonese by examining speech data.
\item Question: are wo (喎) and bo (噃) variant forms?
\item Answer: No
  \begin{quotation}
    {[\ldots]} the two SFPs carry and serve different meanings and functions in modern Hong Kong Cantonese, and thus they are not exactly the same particles and not interchangeable as previously assumed.
    \citep[p21]{Leung:2010}
  \end{quotation}

\end{itemize}

\myslide{References}
\begin{itemize}
\item \cite{Wong:2006}
\item \cite{Leung:2010}
\end{itemize}

\myslide{Hinoki Treebank of Japanese}
 \begin{itemize}
  \item Based on an HPSG grammar of Japanese \com{\jacy}
  \item Parsing dictionary definition sentences \com{\lexeed}
  \item Creating a corpus that can be studied \com{\hinoki}
  \item Creating an ontology that links senses \com{\onto}
  \end{itemize}

\vfill
\begin{center}%\large
  We want to combine \\
  \large \textbf{\jcy{structural}} {\normalsize and}
 \textbf{\ont{lexical}}\\
  semantics
\end{center}
\vfill

\myslide{Creation/Annotation}
\begin{small}
  \begin{itemize}
  \item \jcy{Grammar Development}
    \begin{itemize}
    \item Lexical acquisition from MRDs, Corpora, hand-built
    \item Treebanking (Definition and Example sentences)
    \end{itemize}
  \item \ont{Ontology Development}
    \begin{itemize}
    \item Extracting from MRDs
    \item Boot-strapping from a closed world
    \end{itemize}
  \end{itemize}
\end{small}
\vspace*{-10mm}
\newcommand{\SB}[1]{\psshadowbox[linecolor=yellow]{#1}}
%\newcommand{\SB}[1]{\fbox{#1}}
\begin{scriptsize}
 \begin{psmatrix}
    [name=hinoki]\SB{\begin{tabular}{lrcc}
        \multicolumn{4}{c}{\ul{\emp{Corpus}: \hinoki}}\\
        Type & \# Sents & \jcy{Tree} & \ont{Sense} \\ \hline
        % \multicolumn{2}{c}{構文木, 語義タグ}\\ \hline
        Def. & 81,000 & ○ &   ○   \\
        Ex. &   46,000 & ○ &   ○  \\
        News &   74,000 & △ &   ○   \\
      \end{tabular}}
    &
    [name=lex]\SB{\begin{tabular}{lr}
        \multicolumn{2}{c}{\ul{\emp{Lexicon}: \lexeed}} \\
        \multicolumn{2}{c}{Familiarity, Defs, Exs.} \\ \hline
        Head words & 28,000\\
        \ont{Senses} & 46,000 \\
        % 例文 & 46,000 \\
      \end{tabular}}
    \\[-0.5ex]%
    \SB{\begin{tabular}{lr}
        \multicolumn{2}{c}{\ul{\emp{Grammar}: \jacy (HPSG)}}\\
        \multicolumn{2}{c}{Lex-types, rules, lex-items} \\ \hline
        Lex-items &     37,000\\
        Types &  7,000 \\
        Rules &  114 \\

      \end{tabular}} &

    \SB{\begin{tabular}{lr}
        \multicolumn{2}{c}{\ul{\emp{Meaning}: \onto}}\\
        \multicolumn{2}{c}{links the \lexeed senses}\\ \hline
        relation types &  11\\
        % & \small 上位, 同位, 分野, \ldots \\
        \multicolumn{2}{c}{\smaller \iz{hypernym, meronym,  \ldots}}\\
        relations & 81,300  \\
        % リンク & 語彙大系, WordNet
      \end{tabular}}
  \end{psmatrix}
 \end{scriptsize}
  

\myslide{Usage}
\begin{itemize}
\item Ontology Extraction (also for English)
\item Parse Ranking using Semantics (+3.8\%)
\item Word Sense Disambiguation using Parsing
\item POS tagger training
\end{itemize}

\myslide{References}
\begin{itemize}
\item \cite{Bond:Fujita:Tanaka:2008}
\item \cite{Tanaka:Bond:Baldwin:Fujita:Hashimoto:2007}
\item \cite{Fujita:Bond:Tanaka:Oepen:2010}
\end{itemize}

% \myslide{Redwoods Treebank of English}
% \myslide{Creation/Annotation}
% \myslide{Usage}
% \myslide{References}
% \begin{itemize}
% \item \cite{}
% \item \cite{}
% \end{itemize}

% \myslide{Tatoeba Corpus}
% \myslide{Creation/Annotation}
% \myslide{Usage}
% \myslide{References}
% \begin{itemize}
% \item \cite{}
% \item \cite{}
% \end{itemize}

% \myslide{NICT Multilingual Corpus/Kyoto Corpus}
% \myslide{Creation/Annotation}
% \myslide{Usage}
% \myslide{References}
% \begin{itemize}
% \item \cite{}
% \item \cite{}
% \end{itemize}

% \myslide{NTU Multilingual Corpus}
% \myslide{Creation/Annotation}
% \myslide{Usage}
% \myslide{References}
% \begin{itemize}
% \item \cite{}
% \item \cite{}
% \end{itemize}


% \myslide{Acknowledgments}
% \MyLogo{HG351 (2011)}

% \begin{itemize}
% \item Thanks to Na-Rae Han for 
%   inspiration for some of the slides (from  \textit{LING 2050 Special Topics in Linguistics: Corpus linguistics}, U Penn).
% \item Thanks to Sandra K\"{u}bler for some of the slides from her 
% \textit{RoCoLi\footnote{Romania Computational Linguistics Summer School} Course: Computational Tools for Corpus Linguistics}
% %\item Thanks to Mark Davies (BYU) for the exploration ideas.
% \item Definitions from WordNet 3.0
% \end{itemize}

%%
%% Future
%%

\end{document}


%%% Local Variables: 
%%% coding: utf-8
%%% mode: latex
%%% TeX-PDF-mode: t
%%% TeX-engine: xetex
%%% LaTeX-section-list:  (("myslide" 1))
%%% End: 