\PassOptionsToPackage{xetex}{xcolor}
\PassOptionsToPackage{xetex}{graphicx}
\documentclass[a4paper,landscape,headrule,footrule,xetex]{foils}

%%%
%%% Revision
%%%  * some notes on copyright and licensing
%%%  * style guide
%%%
%%
%%% macros for 2009 Semester 1 HG 803
%%%
\newcommand{\logo}{~}
\newcommand{\header}[3]{%
  \title{\vspace*{-2ex} \large HG3051  Corpus Linquistics
    \\[2ex] \Large  \emp{#2} \\ \emp{#3}}
  \author{\blu{Francis Bond}   \\ 
    \normalsize  \textbf{Division of Linguistics and Multilingual Studies}\\
    \normalsize  \url{http://www3.ntu.edu.sg/home/fcbond/}\\
    \normalsize  \texttt{bond@ieee.org}}
  \MyLogo{HG3051 (2018)}
  \renewcommand{\logo}{#2}
  \hypersetup{
    pdfinfo={
      Author={Francis Bond},
      Title={#1: #2},
      Subject={HG3051: Corpus Linguistics},
      Keywords={Corpus Linguistics},
      License={CC BY 4.0}
    }
  }
  \date{#1 \\ \url{https://github.com/bond-lab/Corpus-Linguistics}}
}

\usepackage{fontenc}
\usepackage{polyglossia}
\setmainlanguage{english}
\setmainfont{TeX Gyre Pagella}
%\setmainfont{Linux Libertine}
%\setmainfont{Charis SIL}
\newfontfamily{\ipafont}{Gentium}
\newcommand{\ipa}[1]{{\ipafont\selectfont #1}}
\usepackage{xeCJK}

\setCJKmainfont{Noto Sans CJK SC}
\setCJKsansfont{Noto Sans CJK SC}



\usepackage{xcolor}
\usepackage{graphicx}
\newcommand{\blu}[1]{\textcolor{blue}{#1}}
\newcommand{\grn}[1]{\textcolor{green}{#1}}
\newcommand{\hide}[1]{\textcolor{white}{#1}}
\newcommand{\emp}[1]{\textcolor{red}{#1}}
\newcommand{\txx}[1]{\textbf{\textcolor{blue}{#1}}}
\newcommand{\lex}[1]{\textbf{\mtcitestyle{#1}}}

\usepackage{pifont}
\renewcommand{\labelitemi}{\textcolor{violet}{\ding{227}}}
\renewcommand{\labelitemii}{\textcolor{purple}{\ding{226}}}

\newcommand{\subhead}[1]{\noindent\textbf{#1}\\[5mm]}

\newcommand{\Bad}{\emp{\raisebox{0.15ex}{\ensuremath{\mathbf{\otimes}}}}}
\newcommand{\bad}{*}

\newcommand{\com}[1]{\hfill \textnormal{(\emp{#1})}}%
\newcommand{\cxm}[1]{\hfill \textnormal{(\txx{#1})}}%
\newcommand{\cmm}[1]{\hfill \textnormal{(#1)}}%

\usepackage{relsize,xspace}
\newcommand{\into}{\ensuremath{\rightarrow}\xspace}
\newcommand{\ent}{\ensuremath{\Rightarrow}\xspace}
\newcommand{\nent}{\ensuremath{\not\Rightarrow}\xspace}
\newcommand{\tot}{\ensuremath{\leftrightarrow}\xspace}
\usepackage{url}
\newcommand{\lurl}[1]{\MyLogo{\url{#1}}}

\usepackage{mygb4e}
\let\eachwordone=\itshape
\newcommand{\lx}[1]{\textbf{\textit{#1}}}

%\usepackage{times}
%\usepackage{nttfoilhead}
\newcommand{\myslide}[1]{\foilhead[-25mm]{\raisebox{12mm}[0mm]{\emp{#1}}}\MyLogo{\logo}}
\newcommand{\myslider}[1]{\rotatefoilhead[-25mm]{\raisebox{12mm}[0mm]{\emp{#1}}}}
%\newcommand{\myslider}[1]{\rotatefoilhead{\raisebox{-8mm}{\emp{#1}}}}

\newcommand{\section}[1]{\myslide{}{\begin{center}\Huge \emp{#1}\end{center}}}



\usepackage[lyons,j,e,k]{mtg2e}
\renewcommand{\mtcitestyle}[1]{\textcolor{teal}{\textsl{#1}}}
%\renewcommand{\mtcitestyle}[1]{\textsl{#1}}
\newcommand{\chn}{\mtciteform}
\newcommand{\cmn}{\mtciteform}
\newcommand{\iz}[1]{\textup{\texttt{\textcolor{blue}{\textbf{#1}}}}}
\newcommand{\rel}[1]{\textsc{\color{blue}{#1}}}
\newcommand{\wn}[3]{\lex{#1}\ensuremath{_{#2:#3}}}
\newcommand{\con}[1]{\textsc{#1}}
\newcommand{\gm}{\textsc}
\usepackage[normalem]{ulem}
\newcommand{\ul}{\uline}
\newcommand{\ull}{\uuline}
\newcommand{\wl}{\uwave}
\newcommand{\vs}{\ensuremath{\Leftrightarrow}~}
\usepackage[hidelinks]{hyperref}
\hypersetup{
     colorlinks,
     linkcolor={blue!50!black},
     citecolor={red!50!black},
     urlcolor={blue!80!black}
}
%%%
%%% Bibliography
%%%
\usepackage{natbib}
%\usepackage{url}
\usepackage{bibentry}
%%% From Tim
\newcommand{\WMngram}[1][]{$n$-gram#1\xspace}
\newcommand{\infers}{$\rightarrow$\xspace}

\usepackage{bibentry}
\renewcommand{\cite}{\bibentry}

\header{Lecture 12}{Review of Corpus Linguistics}{}

\usepackage{pst-node}
\newcommand{\sa}[2]{\rnode{c#1}{\iz{#2}}}%\nodebox{c#1}}

%\usepackage{hieroglf}
\usepackage{wasysym}
%\newcommand{\grn}[1]{\textcolor{PineGreen}{#1}}
\newcommand{\ont}[1]{\textcolor{blue}{#1}}
\newcommand{\jcy}[1]{\textcolor{orange}{#1}}
\newcommand{\lxd}[1]{\textcolor{brown}{#1}}


\newcommand{\psp}[1]{\textcolor{purple}{#1}}
\newcommand{\dtr}[1]{\textcolor{blue}{#1}}
\newcommand{\nam}[1]{\textcolor{blue}{#1}}
\newcommand{\idm}[1]{\textcolor{brown}{#1}}
\newcommand{\gen}[1]{\textcolor{orange}{#1}}
\newcommand{\exl}[1]{\textcolor{orange}{#1}}
\newcommand{\trg}[1]{\textcolor{red}{#1}}


\newcommand{\hinoki}{\grn{Hinoki}\xspace}
\newcommand{\lexeed}{\lxd{Lexeed}\xspace}
\newcommand{\jacy}{\jcy{JACY}\xspace}
\newcommand{\onto}{\ont{Ontology}\xspace}
%\newcommand{\itsdb}{\textsf{[incr tsdb()]}\xspace}
\newcommand{\GT}{Goi-Taikei\xspace}


\begin{document}
\bibliographystyle{apalike}
\nobibliography{abb,mtg,nlp,ling}
\maketitle


\myslide{Overview}

\begin{itemize} \addtolength{\itemsep}{-0.3\itemsep}
%\item What can we do with Corpora?
\item Markup and Annotation 
\item Using Corpora: Regular Expressions
\item Multimodal and Multilingual Corpora 
\item Collocation, Frequency, Corpus Statistics 
\item DIY Corpora, Corpus Tools, Processing Raw Text 
\item Case studies: Lexical, Grammatical, Contrastive, Diachronic
\item Corpora and Language Engineering 
\item Representativeness and Balance
\item Copyright and Licensing
\end{itemize}


%%% 
%%% this changes each year, so keep separate
%%%
\include{schedule}

\section{More SQL \\ Creating; Inserting; \\ Updating and Deleting}

\myslide{How to create a Table}
\MyLogo{There are more types, if you must know, google them}
\begin{verbatim}
CREATE TABLE database_name.table_name(
   column1 datatype  PRIMARY KEY(one or more columns),
   column2 datatype,
   column3 datatype,
   .....
   columnN datatype,
);
\end{verbatim}

Each column should have a \txx{datatype}

\begin{tabular}{ll}
  TEXT & 	A text string, stored using the database encoding\\
  INTEGER	& Signed integer (or INT)\\
  REAL	& Floating point number \\
  CHAR(N)  & String of N characters padded with spaces \\
  VARCHAR(N)  & String of N characters
\end{tabular}

\texttt{sqlite} is very forgiving, you can store any data type in any column. 

\myslide{For example: thw word table}

\begin{verbatim}
CREATE TABLE word (
-- store words, with POS and lemma
-- start and end in the corresponding sentence (cfrom, cto)
       sid INTEGER,     -- sentence ID
       wid INTEGER,     -- wid (should be consecutive)
       word TEXT,       -- surface form of the word
       pos TEXT,        -- part of speech
       lemma TEXT,      -- lemma (true-cased)
       cfrom INTEGER,   -- start position 
       cto INTEGER,     -- end position  
       comment TEXT,
       PRIMARY KEY (sid, wid),
       FOREIGN KEY(sid) REFERENCES sent(sid)
       );
\end{verbatim}
\myslide{PRIMARY KEYS}
\MyLogo{If you don't define one, nothing bad happens}
\begin{itemize}
\item The PRIMARY KEY constraint uniquely identifies each record in a database table.
\item Primary keys must contain UNIQUE values.
\item A primary key column cannot contain NULL values.
\item Each table can have only ONE primary key.
\item Most tables should have a primary key
\end{itemize}

\myslide{You can show it with \texttt{.tables} or \texttt{.schema}}

\begin{verbatim}
sqlite>.tables
sent word concept ...

sqlite>.schema word
CREATE TABLE word (
	sid INTEGER,
	wid INTEGER,
	word TEXT,
	pos TEXT,
	lemma TEXT,
	cfrom INTEGER,
	cto INTEGER,
	comment TEXT,
        PRIMARY KEY (sid, wid),
        FOREIGN KEY(sid) REFERENCES sent(sid) );
\end{verbatim}

\myslide{Inserting Information}

\begin{verbatim}
INSERT INTO word (sid, wid, word, pos, lemma)  
  VALUES (1, 0, "The", "DT", "the");
INSERT INTO word (sid, wid, word, pos, lemma)  
  VALUES (1, 1, "Adventure", "NNS", "ADVENTURE");
INSERT INTO word (sid, wid, word, pos, lemma)  
  VALUES (1, 2, "of", "PP", "of");
\end{verbatim}

\myslide{Upating Information}

\begin{verbatim}
UPDATE word SET lemma='adventure'
  WHERE sid=1 AND wid=1;
\end{verbatim}

or

\begin{verbatim}
UPDATE word SET lemma='adventure'
  WHERE lemma='ADVENTURE';
\end{verbatim}

\emp{Everything that matches the condition gets updated}

Best to check with a SELECT first:
\begin{verbatim}
SELECT * FROM word 
  WHERE lemma='ADVENTURE';
\end{verbatim}

\myslide{Deleting Information}

Be very, very careful:
\begin{verbatim}
DELETE FROM table_name
  WHERE [condition];
\end{verbatim}

\myslide{Dates and times}
\MyLogo{And much more: \url{http://www.tutorialspoint.com/sqlite/sqlite_date_time.htm}}
\begin{tabular}{ll}
Time String  & Example \\ \hline
YYYY-MM-DD & 2010-12-30 \\
YYYY-MM-DD HH:MM & 2010-12-30 12:10 \\
YYYY-MM-DD HH:MM:SS.SSS & 2010-12-30 12:10:04.100 \\
MM-DD-YYYY HH:MM & 30-12-2010 12:10 \\
HH:MM & 12:10 \\
YYYY-MM-DDTHH:MM & 2010-12-30 12:10 \\
HH:MM:SS & 12:10:01 \\
now & 2015-04-15
\end{tabular}
\begin{verbatim}
sqlite> SELECT date('now');
2015-04-15
sqlite> SELECT date('now', '+1 months');
2015-05-15
sqlite> SELECT date('now', 'start of month');
2015-05-01
\end{verbatim}

\myslide{Task}

\begin{itemize}
\item Create a new table in your database
\item Add three entries
\item Update two
\item Delete one
\end{itemize}

\myslide{Make a bigram Table}

\begin{verbatim}
create TABLE bigram (sid INT, wid INT, bigram TEXT);

INSERT INTO bigram (sid, wid, bigram) 
  SELECT a.sid, a.wid, a.lemma || ' ' || b.lemma  
    FROM word AS a JOIN word AS b 
    ON a.sid=b.sid AND a.wid = b.wid-1 
  LIMIT 5;
\end{verbatim}

The result:

\begin{verbatim}
sqlite> SELECT sid, wid, bigram FROM bigram;
60000	0	prime minister_tomiichi_murayama
60000	1	minister_tomiichi_murayama on
60000	2	on the
60000	3	the 28
60000	4	28 hold
\end{verbatim}

\myslide{Trading SPACE for TIME}
\MyLogo{There are whole courses on this}
\begin{itemize}
\item Storing bigrams makes the DB bigger
\item But you can manipulate them quickly
\item For large tables, you can also \txx{INDEX} them
\begin{verbatim}
CREATE INDEX word_idx
on word (lemma, pos);
\end{verbatim}
\item This allows you to query \texttt{word} or \texttt{word+pos} much faster
\item Use indexes for big tables you search often but don't update much
\item Indexes can double the size of your database
  \begin{itemize}
  \item But speed big searches up from hours to seconds
  \end{itemize}


\end{itemize}

\myslide{Batch Import}

\begin{itemize}
\item You can input well formatted data using \texttt{sqliteman} or similar
  \begin{itemize}
  \item define the column separator ':' or '$\|$' or TAB or ',' or \ldots
  \item or load from spreadsheet
  \end{itemize}
\item Or through some program
  \begin{itemize}
  \item Learn more in HG2051 \textit{Language and the Computer}
  \end{itemize}
\end{itemize}



\section{Revision}
\MyLogo{}


\myslide{The goal of this course}

\begin{center}
  \LARGE 
Master the uses of text corpora 
\\ in linguistics research and applications.
\end{center}
\begin{itemize}
\item Selecting text
\item Marking up extra information
\item The range of existing corpora
\item How to build your own corpus
\item Using corpora to test linguistic hypotheses
\item Using corpora to train language tools
\end{itemize}


\myslide{What did you learn?}

You should be able to:

\begin{itemize}
\item Understand the uses of text corpora in language research
  \\ Be able to manipulate them with simple tools
\item Use a concordance program to extract data from a corpus
\item Design and build a corpus for some task
\begin{itemize}
\item considering representativeness, balance and legal issues
\item  as well as usability and accuracy
\end{itemize}
\item Understand how to analyse corpus data through basic statistical methods
\item Understand the issues involved in using data for NLP
\end{itemize}



\myslide{Reflection}
\begin{itemize}
\item What was the most surprising thing in this class?
\item What do you think is most likely wrong?
\item What do you think is the coolest result/corpus?
\item What do you think you’re most likely to
remember?
\item How do you think this course will influence you as a linguist/specialist?
\item What (if anything) did you hope to learn that you didn't?
\end{itemize}


\end{document}


%%% Local Variables: 
%%% coding: utf-8
%%% mode: latex
%%% TeX-PDF-mode: t
%%% TeX-engine: xetex
%%% LaTeX-section-list:  (("myslide" 1))
%%% End: 

