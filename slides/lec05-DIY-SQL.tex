\PassOptionsToPackage{xetex}{xcolor}
\PassOptionsToPackage{xetex}{graphicx}
\documentclass[a4paper,landscape,headrule,footrule,xetex]{foils}


%%
%%% macros for 2009 Semester 1 HG 803
%%%
\newcommand{\logo}{~}
\newcommand{\header}[3]{%
  \title{\vspace*{-2ex} \large HG3051  Corpus Linquistics
    \\[2ex] \Large  \emp{#2} \\ \emp{#3}}
  \author{\blu{Francis Bond}   \\ 
    \normalsize  \textbf{Division of Linguistics and Multilingual Studies}\\
    \normalsize  \url{http://www3.ntu.edu.sg/home/fcbond/}\\
    \normalsize  \texttt{bond@ieee.org}}
  \MyLogo{HG3051 (2018)}
  \renewcommand{\logo}{#2}
  \hypersetup{
    pdfinfo={
      Author={Francis Bond},
      Title={#1: #2},
      Subject={HG3051: Corpus Linguistics},
      Keywords={Corpus Linguistics},
      License={CC BY 4.0}
    }
  }
  \date{#1 \\ \url{https://github.com/bond-lab/Corpus-Linguistics}}
}

\usepackage{fontenc}
\usepackage{polyglossia}
\setmainlanguage{english}
\setmainfont{TeX Gyre Pagella}
%\setmainfont{Linux Libertine}
%\setmainfont{Charis SIL}
\newfontfamily{\ipafont}{Gentium}
\newcommand{\ipa}[1]{{\ipafont\selectfont #1}}
\usepackage{xeCJK}

\setCJKmainfont{Noto Sans CJK SC}
\setCJKsansfont{Noto Sans CJK SC}



\usepackage{xcolor}
\usepackage{graphicx}
\newcommand{\blu}[1]{\textcolor{blue}{#1}}
\newcommand{\grn}[1]{\textcolor{green}{#1}}
\newcommand{\hide}[1]{\textcolor{white}{#1}}
\newcommand{\emp}[1]{\textcolor{red}{#1}}
\newcommand{\txx}[1]{\textbf{\textcolor{blue}{#1}}}
\newcommand{\lex}[1]{\textbf{\mtcitestyle{#1}}}

\usepackage{pifont}
\renewcommand{\labelitemi}{\textcolor{violet}{\ding{227}}}
\renewcommand{\labelitemii}{\textcolor{purple}{\ding{226}}}

\newcommand{\subhead}[1]{\noindent\textbf{#1}\\[5mm]}

\newcommand{\Bad}{\emp{\raisebox{0.15ex}{\ensuremath{\mathbf{\otimes}}}}}
\newcommand{\bad}{*}

\newcommand{\com}[1]{\hfill \textnormal{(\emp{#1})}}%
\newcommand{\cxm}[1]{\hfill \textnormal{(\txx{#1})}}%
\newcommand{\cmm}[1]{\hfill \textnormal{(#1)}}%

\usepackage{relsize,xspace}
\newcommand{\into}{\ensuremath{\rightarrow}\xspace}
\newcommand{\ent}{\ensuremath{\Rightarrow}\xspace}
\newcommand{\nent}{\ensuremath{\not\Rightarrow}\xspace}
\newcommand{\tot}{\ensuremath{\leftrightarrow}\xspace}
\usepackage{url}
\newcommand{\lurl}[1]{\MyLogo{\url{#1}}}

\usepackage{mygb4e}
\let\eachwordone=\itshape
\newcommand{\lx}[1]{\textbf{\textit{#1}}}

%\usepackage{times}
%\usepackage{nttfoilhead}
\newcommand{\myslide}[1]{\foilhead[-25mm]{\raisebox{12mm}[0mm]{\emp{#1}}}\MyLogo{\logo}}
\newcommand{\myslider}[1]{\rotatefoilhead[-25mm]{\raisebox{12mm}[0mm]{\emp{#1}}}}
%\newcommand{\myslider}[1]{\rotatefoilhead{\raisebox{-8mm}{\emp{#1}}}}

\newcommand{\section}[1]{\myslide{}{\begin{center}\Huge \emp{#1}\end{center}}}



\usepackage[lyons,j,e,k]{mtg2e}
\renewcommand{\mtcitestyle}[1]{\textcolor{teal}{\textsl{#1}}}
%\renewcommand{\mtcitestyle}[1]{\textsl{#1}}
\newcommand{\chn}{\mtciteform}
\newcommand{\cmn}{\mtciteform}
\newcommand{\iz}[1]{\textup{\texttt{\textcolor{blue}{\textbf{#1}}}}}
\newcommand{\rel}[1]{\textsc{\color{blue}{#1}}}
\newcommand{\wn}[3]{\lex{#1}\ensuremath{_{#2:#3}}}
\newcommand{\con}[1]{\textsc{#1}}
\newcommand{\gm}{\textsc}
\usepackage[normalem]{ulem}
\newcommand{\ul}{\uline}
\newcommand{\ull}{\uuline}
\newcommand{\wl}{\uwave}
\newcommand{\vs}{\ensuremath{\Leftrightarrow}~}
\usepackage[hidelinks]{hyperref}
\hypersetup{
     colorlinks,
     linkcolor={blue!50!black},
     citecolor={red!50!black},
     urlcolor={blue!80!black}
}
%%%
%%% Bibliography
%%%
\usepackage{natbib}
%\usepackage{url}
\usepackage{bibentry}
%%% From Tim
\newcommand{\WMngram}[1][]{$n$-gram#1\xspace}
\newcommand{\infers}{$\rightarrow$\xspace}

\usepackage{bibentry}
\renewcommand{\cite}{\bibentry}

\header{Lecture 6}{DIY Corpora,  Processing Raw Text, SQL}{}

\usepackage{pst-node}
\newcommand{\sa}[2]{\rnode{c#1}{\iz{#2}}}%\nodebox{c#1}}

%\usepackage{hieroglf}
\usepackage{wasysym}
%\newcommand{\grn}[1]{\textcolor{PineGreen}{#1}}
\newcommand{\ont}[1]{\textcolor{blue}{#1}}
\newcommand{\jcy}[1]{\textcolor{orange}{#1}}
\newcommand{\lxd}[1]{\textcolor{brown}{#1}}

\newcommand{\hinoki}{\grn{Hinoki}\xspace}
\newcommand{\lexeed}{\lxd{Lexeed}\xspace}
\newcommand{\jacy}{\jcy{JACY}\xspace}
\newcommand{\onto}{\ont{Ontology}\xspace}
%\newcommand{\itsdb}{\textsf{[incr tsdb()]}\xspace}
\newcommand{\GT}{Goi-Taikei\xspace}


\begin{document}
\bibliographystyle{apalike}
\nobibliography{abb,mtg,nlp,ling}
\maketitle


\myslide{Overview}

\begin{itemize} 
\item DIY Corpora
\item Processing Raw Text
\item Structured Query Language
\end{itemize}
%%% 
%%% this changes each year, so keep separate
%%%
\include{schedule}


\section{DIY Corpora}

\myslide{Why build your own?}

\begin{enumerate}
\item Decide what you want to study
\\ see if you can do it with existing resources
\\ No \frownie
\item Collect data that fits your needs
  \begin{itemize}
  \item Speech (expensive)
  \item Text (easy to get, hard to do legally)
  \end{itemize}
\item Process it
  \begin{itemize}
  \item Clean up
  \item Mark up (what do we know about the data originally)
  \item Annotation (what will we add to it)
  \end{itemize}
\end{enumerate}

\myslide{Collecting Text}

Some ways to collect text:

\begin{itemize}\addtolength{\itemsep}{-1ex}
\item Word-processed texts: save as a text only file
\item Keyboard entry: speech transcription, students' handwritten essays, etc.
\item Scanning: copyright protected novels, latest magazines, etc.
\item CD-ROM: newspaper, encyclopaedia, ICAME, etc.
\item Internet resources: email, chat, public documents, newspapers, magazines etc.
\item  Text archives: copyright free (old) novels, essays, etc.
\item Copying from a large corpus: e.g. using sections of the BNC
\end{itemize}

\myslide{Recent Trends}
\MyLogo{Moving to Open Data}
\begin{itemize}
\item Government Documents (proceedings, publications)
\item Open source documents (manuals, wikipedia)
\item Social Media:  Blogs and twitter
\end{itemize}

\myslide{Web as Corpus}
\myslide{Two Approaches to using the Web as a Corpus}
\begin{itemize}
\item \blu{Direct Query}: Search Engine as Query tool and WWW as corpus?
\\  (Objection: Results are not reliable)
\begin{itemize}
\item Population and exact hit counts are unknown → no statistics
possible.
\item Indexing does not allow to draw conclusions on the data.
\item[\Bad] Google is missing functionalities that linguists /
lexicographers would like to have.
\end{itemize}
\item \blu{Web Sample}: Use search engine to download data from the
net and build a corpus from it.
\begin{itemize}
\item known size and exact hit counts → statistics possible.
\item people can draw conclusions over the included text types.
\item (limited) control over the content.
\item[\Bad] sparser data
\end{itemize}
\end{itemize}

\myslide{Direct Query}
\begin{itemize}
\item Accessible through search engines (Google API, Yahoo API, Scripts)

\item Document counts are shown to correlate directly with ``real''
  frequencies (Keller 2003), so search engines can help - but...
  \begin{itemize}
  \item lots of repetitions of the same text (not representative)
  \item very limited query precision (no upper/lower case, no punctuation...)
  \item only estimated counts, often hard to reproduce exactly
  \item different queries give wildly different numbers
  \end{itemize}
\end{itemize}

\myslide{Web Sample}
\MyLogo{}
\begin{itemize}
\item Extracting and filtering web documents to create linguistically
  annotated corpora (Kilgarriff 2006)
  \begin{itemize}
  \item gather documents for different topics (balance!)
  \item exclude documents which cannot be preprocessed with available
    tools (here taggers and lemmatizers)
  \item exclude documents which seem irrelevant for a corpus (too short or
    too long, word lists,...)
  \item do this for several languages and make the corpora available
  \end{itemize}
\end{itemize}


\myslide{Building Internet Corpora: Outline}
\MyLogo{\url{http://corpus.leeds.ac.uk/internet.html}}

\begin{enumerate}
\item Select Seed Words (500)
\item Combine to form multiple queries (6,000)
\item Query a search engine and retrieve the URLs (50,000)
\item Download the files from the URLS (100,000,000 words)
\item Postprocess the data (encoding; cleanup; tagging and parsing)
\end{enumerate}

Sharoff, S (2006) Creating general-purpose corpora using automated search engine queries. In M. Baroni, S. Bernardini (eds.) WaCky! Working papers on the Web as Corpus, Bologna, 2006.



\myslide{Post-processing}
\begin{itemize}
\item Filter documents by size
  \begin{itemize}
  \item Small documents ($<5KB$) contain very little real text
  \item Large documents ($>200KB$) tend to be indices, catalogues, lists, etc.
  \end{itemize}
\item Remove perfect duplicates
  \begin{itemize}
  \item Actually, removed both the original \& the duplicate:
  \item[\ldots] tend to be warning messages 
  \end{itemize}
\end{itemize}

\myslide{Boilerplate stripping}
\begin{itemize}
\item \blu{Boilerplate}: HTML markup, javascript, other non-linguistic
  material
\item Removing boilerplate information is crucial to obtaining
  linguistic data only
  \begin{itemize}
  \item Content-rich sections of a document will have a low 
    html tag density
  \item Boilerplate sections have a wealth of html
  \item This heuristic is ``relatively independent of language and
    crawling strategy''
  \end{itemize}
\item If a text does not have enough function words, it is likely
non-linguistic material (e.g., a list)
\begin{itemize}
\item Require at least 10 function word types and 30 tokens on a page
\item[\ldots]  which must make up at least 25\% of the total
words
\end{itemize}
\end{itemize}

\myslide{Near-duplicate detection}
\begin{itemize}
\item Take \blu{fingerprints}' of a fixed number of randomly-selected $n$-grams (ignoring function words)
  \begin{itemize}
  \item e.g., extract 25 5-grams from each document
  \end{itemize}
\item Near-duplicates have a high overlap
 \begin{itemize}
 \item e.g., at least 2 5-grams in common
 \end{itemize}
\end{itemize}

%%% FIXME define ngrams

\myslide{Linguistic Post-processing}


\begin{itemize}
\item Prepare the  data  for searching:
  \begin{itemize}
  \item Run a POS tagger over it
  \item Clean the documents further, using POS tags
    \begin{itemize}
    \item Where the POS tag distribution is unusual,
    \item[\ldots] perform another round of anomalous document
finding
\item Look for problematic (erroneous) POS tags and
remove those documents
\item Use cues such as number of unrecognized words,
proporition of words with upper-case initial letters, \ldots
\end{itemize}
\end{itemize}
\item Index the document by word, POS and lemma
\end{itemize}


\myslide{Internet Corpora Summary}
\MyLogo{}
 
\begin{itemize}
\item The web can be used as a corpus
  \begin{itemize}
  \item Direct access
    \begin{itemize}
    \item Fast and convenient
    \item Huge amounts of data
    \item[\Bad] unreliable counts 
    \end{itemize}
  \item Web sample
    \begin{itemize}
    \item Control over the sample
    \item Some setup costs (semi-automated)
    \item[\Bad] Less data 
    \end{itemize}
  \end{itemize}
\item Richer data than a compiled corpus
\item[\Bad] Less balanced, less markup
\end{itemize}



% \myslide{Corpus Tools}

% \begin{itemize}
% \item BNY: 
% \item Sketch Engine \url{https://trac.sketchengine.co.uk/}
% \item Web as Corpus: \url{http://webascorpus.org/searchwc.html}
% \item Internet Corpora: \url{http://corpus.leeds.ac.uk/internet.html}
% \item Google Ngrams: \url{http://ngrams.googlelabs.com/}
% \item Natural Language Tool Kit: \url{http://www.nltk.org/}
% \end{itemize}

\section{Processing Raw Text}

\myslide{Language Identification}

\begin{itemize}

\item Given a document and a list of possible languages, in
  what language was the document written? (e.g.\ English, German, Japanese, Uyghur, ...)
\item Language identification provides us with the means to
  automatically ``discover'' web data to convert into a corpus over
  which to learn linguistic (lexical) properties
\item Main Approaches
\begin{itemize}
\item Linguistically-grounded methods
  \begin{itemize}
  \item Diacritics/Characters
  \item Character $n$-grams
  \item Stop words
  \end{itemize}
\item Statistical Methods
\item Context (under \url{.jp} or \url{.ko}?)
\end{itemize}


\end{itemize}

\myslide{Normalization}

\begin{itemize}
\item Extracting text from various documents
\item Segmenting continuous text
\item Number Normalization:  
  \eng{\$700K}, \eng{\$700,000}, \eng{0.7 million dollars},  \ldots
\item Date Normalization:   \eng{2000AD, 1421AH,  Heisei 12, \ldots} 
\item Stripping stop words
\item Lemmatization:  \textit{produces}   \infers \textit{produce}
\item Stemming:  \textit{producer}   \infers \textit{produc}; \textit{produces}   \infers \textit{produc}
\item Decompounding: \textit{zonnecel}  \infers \textit{zon cel}
\end{itemize}


\section{Relational Databases \\ and SQL}

\myslide{Many corpora are stored as Relational Data-Bases}

\begin{itemize}
\item Data is stored in \txx{Table}s (or relations)
  \begin{itemize}
  \item \txx{attribute}s are  \txx{column}s
  \item \txx{record}s are \txx{row}s
  \end{itemize}
    \item Tables can be joined together
  \begin{itemize}
  \item If they share a common column
  \end{itemize}
\item Data can be retrieved with \txx{Queries}
  \begin{itemize}
  \item Properly written these can be very fast
  \end{itemize}
\end{itemize}
\txx{Structured Query Language} (SQL) is a special-purpose
programming language designed for managing data held in a relational
database management system (RDBMS)


\myslide{Example tables}

\begin{tabular}{lllll}
  \multicolumn{2}{c}{TABLE: sent} \\
  sid & sent \\ \hline
  INTEGER & STRING \\ \hline
  1   & Does this work? \\
  2   & I hope so. \\
\end{tabular}


\begin{tabular}{lllll}
  \multicolumn{5}{c}{TABLE: word} \\
  sid & wid & word & pos& lemma \\ \hline
  INTEGER &INTEGER & STRING& STRING & STRING \\ \hline
  1   &  1 & Does & VBZ & do \\
  1   &  2 & this & DT & this \\
  1   &  3 & work & NN & work \\
  1   &  4 & ?    & . & ? \\
  2   &  1 & I & PRN & i \\
  \ldots
\end{tabular}


\myslide{Select-From-Where}

\begin{itemize} \addtolength{\itemsep}{-1ex}
\item The simplest query
  \begin{itemize}
  \item SELECT desired attributes (columns)
  \item FROM one or more tables
  \item WHERE condition applies about the records  in the tables
  \end{itemize}
\item What lemmas are there associated with the word \eng{does}?
\begin{verbatim}
SELECT word, lemma
FROM word
WHERE word = 'does'
\end{verbatim}

  \begin{tabular}{ll}
    \textbf{word} & \textbf{lemma} \\ \hline
    does & do \\
    does & do \\
    does & doe \\
    \ldots
  \end{tabular}
\end{itemize}

\myslide{How does it work?}

\begin{itemize}
\item Begin with the table in the FROM clause.
\item Apply the selection indicated by the WHERE clause.
\item Select only those parts indicated by the SELECT clause.
\end{itemize}


Note: each query ends with a semicolon ``;'', not shown in the example

\myslide{* in SELECT}

\begin{itemize}
\item When there is one relation in the FROM clause, * in the SELECT clause stands for “all attributes of this relation.”
\item What information is there associated with the word \eng{does}?
\begin{verbatim}
SELECT *
FROM word
WHERE word = 'does'
\end{verbatim}
  \begin{tabular}{lllll}
   \textbf{sid} & \textbf{wid} &  \textbf{word} & \textbf{pos} & \textbf{lemma} \ldots \\ \hline
    10 & 1 & does & VBZ & do \\
    11 & 7 & does & VBZ & do \\
    14 & 4 & does & NNS & doe \\
    \ldots
  \end{tabular}
\end{itemize}

\myslide{You can rename things}

\begin{itemize}
\item What information is there associated with the word \eng{does}?
\begin{verbatim}
SELECT word AS surface, lemma AS indexform, pos
FROM word
WHERE word = 'does'
\end{verbatim}
  \begin{tabular}{lllll}
    \textbf{surface}  & \textbf{indexform}  &  \textbf{pos} \\ \hline
    does & do  & VBZ \\
    does & do  & VBZ \\
    does & doe & NNS \\
    \ldots
  \end{tabular}
\end{itemize}

\myslide{You can sort things}

\begin{itemize}\addtolength{\itemsep}{-1ex}
\item What information is there associated with the word \eng{does}, sorted by POS?
\begin{verbatim}
SELECT word AS surface, lemma AS indexform, pos
FROM word
WHERE word = 'does'
ORDER BY pos DESC
\end{verbatim}
  \begin{tabular}{lllll}
    \textbf{surface}  & \textbf{indexform}  &  \textbf{pos} \\ \hline
    does & do  & VBZ \\
    does & do  & VBZ \\
    does & doe & NNS \\
    \ldots
  \end{tabular}
\item You can specify the order with \texttt{DESC} or \texttt{ASC} (default)
\item You can order by multiple things: \texttt{ORDER BY pos, LENGTH(word)}
\end{itemize}

\myslide{And add new things}

\begin{itemize}\addtolength{\itemsep}{-1ex}
\item What is the lemma, its length and pos associated with the word \eng{does}?
\begin{verbatim}
SELECT  lemma, LENGTH(lemma) AS length, pos
FROM word
WHERE word = 'does'
\end{verbatim}
  \begin{tabular}{lllll}
    \textbf{lemma}  & \textbf{length}  &  \textbf{pos} \\ \hline
    do  & 2 &  VBZ \\
    do  & 2 &  VBZ \\
    doe & 3 &  NNS \\
    \ldots
  \end{tabular}
\item strings:  \texttt{LENGTH(), LOWER(), UPPER(), TRIM(), LTRIM(),
    RTRIM()}
  \\ \texttt{SUBSTR(str,from, len), REPLACE(str, from, into)}
\item numbers:  \texttt{ROUND(num,digits=0), ABS(num)}
\end{itemize}

\myslide{Task}

\begin{itemize}
\item Download \texttt{eng.db} and \texttt{cmn.db}
\item Start the sqlitebrowser
\item Open \eng{eng.db}
  \begin{itemize}
  \item Find all surface forms of \lex{leave}
  \item Find all surface forms of \lex{leave}, and show their length
  \item Find all surface forms of \lex{leave}, ordered from longest to shortest
  \end{itemize}
\end{itemize}

\begin{enumerate}
\item Open the database
\item Look at the database structure
\item Execute SQL
\end{enumerate}

\myslide{You can have complex conditions and limits}

\begin{itemize}
\item Show 5 words that are not the same as their lemmas:
\begin{verbatim}
SELECT  word, lemma, pos
FROM word
WHERE word != lemma
LIMIT 5
\end{verbatim}
  \begin{tabular}{lllll}
    \textbf{word}  & \textbf{lemma}  &  \textbf{pos} \\ \hline
    does  & do &  VBZ \\
    held  & hold &  VBN \\
    does & does &  NNS \\
    Holmes & holmes & NNP \\
    Holmes & holmes & NNP \\
    \ldots
  \end{tabular}
\end{itemize}

\myslide{You can even have simple regular expressions}

\begin{itemize}
\item Show 5 words that include the string \texttt{dog}:
\begin{verbatim}
SELECT  word, lemma, pos
FROM word
WHERE word GLOB '*dog*'
LIMIT 5
\end{verbatim}
  \begin{tabular}{lllll}
    \textbf{word}  & \textbf{lemma}  &  \textbf{pos} \\ \hline
dog-cart   & dog-cart   &  NN      \\  
dog        & dog        &  NN    \\    
dog-       & dog        &  NN   \\     
dog-cart   & dog-cart   &  NN   \\     
dogged     & dog        &  VBD   \\
    \ldots
  \end{tabular}
\end{itemize}

\myslide{Or slightly complicated ones}

\begin{itemize}
\item Show 5 words starting with dog or Dog whose part of speech is not a noun
\begin{verbatim}
SELECT  word, lemma, pos
FROM word
WHERE word GLOB '[dD]og*' AND pos NOT GLOB 'N*'
LIMIT 5
\end{verbatim}
  \begin{tabular}{lllll}
    \textbf{word}  & \textbf{lemma}  &  \textbf{pos} \\ \hline
    dogged  & dog &  VBN \\
  \end{tabular}
\begin{itemize}
\item[*] matches anything
\item[?] matches one or one of anything
\item[{[\ ]}] matches all characters listed (and allows
  \textasciicircum\ for negation and - for ranges)
  \\ e.g., [a-z], [\textasciicircum{}H], 
\end{itemize}
\end{itemize}

\myslide{You can aggregate results}

\begin{itemize}
\item Tell me more about the words
\begin{verbatim}
SELECT COUNT(word), COUNT(DISTINCT word),
       MIN(LENGTH(word)), 
       MAX(LENGTH(word)), AVG(LENGTH(word))
FROM word
\end{verbatim}
  \begin{tabular}{lllll}
    \textbf{count}  & \textbf{distinct} &\textbf{min}  &  \textbf{max} & \textbf{avg} \\ \hline
    77820 & 9485 & 1 & 86 & 4.437
  \end{tabular}
\item  \texttt{MIN(num), AVG(num), MAX(num)} are calculated over the whole set
\item \texttt{DISTINCT} eliminates duplicate records thus fetches only unique records
\end{itemize}

\myslide{You can group things}
\MyLogo{A random word (the first it finds) will be inserted}
\begin{itemize}
\item Tell me more about the words grouped into parts-of-speech
\begin{verbatim}
SELECT pos, word,
       count(word), count(DISTINCT word),
       MIN(LENGTH(word)), 
       MAX(LENGTH(word)), AVG(LENGTH(word))
FROM word
GROUP BY pos
\end{verbatim}
  \begin{tabular}{llrrrrr}
    \textbf{pos} & \textbf{word} & \textbf{count} & \textbf{distinct}  
  & \textbf{min}  &  \textbf{max} & \textbf{avg} \\ \hline
  CC & and  & 2264 & 13 & 2 & 7 & 2.98 \\
  DT & the  & 8256 & 38 & 1 & 7 & 2.74 \\
  EX & There&  243 &  2 & 5 & 5 & 5.00  \\
\ldots
  \end{tabular}
\end{itemize}


\myslide{Task}

\begin{itemize}
\item Which POS has the greatest difference between lemma and word length?
\item Which preposition has a lemma not equal to its surface form?
\item Find the number of words in each POS and sort from most to least
  frequent (for both English and Chinese)
\end{itemize}

\myslide{You can have a list in your condition}

\begin{itemize}
\item Tell me more about common nouns
\begin{verbatim}
SELECT count(word), count(DISTINCT word),
       MIN(LENGTH(word)), 
       MAX(LENGTH(word)), AVG(LENGTH(word))
FROM word
WHERE pos IN ('NN', 'NNS')
\end{verbatim}
  \begin{tabular}{lllll}
    \textbf{count}  & \textbf{distinct} &\textbf{min}  &  \textbf{max} & \textbf{avg} \\ \hline
    14457 & 3593 & 1 & 21 & 6.74
  \end{tabular}
\item  \texttt{MIN(num), AVG(num), MAX(num)} are calculated over the whole set
\item \texttt{DISTINCT} eliminates duplicate records thus fetches only unique records
\end{itemize}

\myslide{This list can be the result of a select!}

\begin{itemize}
\item Tell me more about frequent common nouns
\begin{verbatim}
SELECT count(word), count(DISTINCT word),
       MIN(LENGTH(word)), 
       MAX(LENGTH(word)), AVG(LENGTH(word))
FROM word
WHERE word 
IN (SELECT word
    FROM word
    WHERE POS in ('NN', 'NNS') 
    GROUP BY word
    ORDER BY COUNT(word) DESC
    LIMIT 10)

\end{verbatim}
  \begin{tabular}{lllll}
    \textbf{count}  & \textbf{distinct} &\textbf{min}  &  \textbf{max} & \textbf{avg} \\ \hline
    1096 & 10 & 3 & 10 & 5.13
  \end{tabular}
\item First find the ten most common words, then do things to them
  \begin{itemize}
  \item the trick is to write simple queries first, and then combine them
  \end{itemize}
\item Note that more common words are shorter (as we would expect)
\end{itemize}
\begin{center}
  Now try to do some corpus queries yourself!
\end{center}


\myslide{You can Create Tables}

You tell the database what the data will look like:
\begin{verbatim}
CREATE TABLE word (
    sid INTEGER,
    wid INTEGER,
    word TEXT,
    pos TEXT,
    lemma TEXT,
    cfrom INTEGER,
    cto INTEGER,
    comment TEXT,
    usrname TEXT,			
    PRIMARY KEY (sid, wid),
    FOREIGN KEY(sid) REFERENCES sent(sid)
    );
\end{verbatim}

\myslide{You can Drop Tables}
You tell the database what the data will look like:
\begin{verbatim}
DROP TABLE word;
\end{verbatim}

But be careful, you can't get it back.

\myslide{You can Insert Data}

\begin{verbatim}
INSERT INTO word (sid, wid, word, lemma, pos)  
VALUES (1, 1, 'The', 'the', 'DT');
\end{verbatim}

Normally you would add data using a program, or read it in from some
other file, \ldots

\myslide{You can Update Data}

\begin{verbatim}
UPDATE word SET lemma='a', word='A'
WHERE sid=1,wid=1;
\end{verbatim}

This changes the value permanently for all rows that match the
condition.

\begin{verbatim}
UPDATE word SET pos='ART'
WHERE pos = 'DT';
\end{verbatim}

You can change a lot at a time.

\myslide{You can Delete Data}

\begin{verbatim}
DELETE FROM word WHERE word='gannet';
\end{verbatim}

But be careful, you can't get it back.

\myslide{You can Index Data}

Indexes are special tables that the database can use to speed up data
retrieval.  An index is a pointer to data in a table, think of it as
index in the back of a book.  An index helps to speed up SELECT
queries and WHERE clauses, but it slows down data input, with the
UPDATE and the INSERT statements. Indexes can be created or dropped
with no effect on the data.

\begin{verbatim}
CREATE INDEX word_word_idx ON word (word);
CREATE INDEX word_lemma_idx ON word (lemma);
\end{verbatim}

This makes it possible to look up words and lemmas very fast, but makes the
database bigger.  You normally add them after you have added the
data.  Whether they speed things up or not is an empirical question,
and should be tested.

\myslide{SQL and NLP}


\begin{itemize}
\item RDMS and SQL are the backbone of most data storage
\item There are many implementations:
  \begin{itemize}
  \item SQLite, PostgreSQL, ORACLE, MySQL, MS SQL Server, \ldots
  \item typically share the same core
  \item may have different extensions
  \end{itemize}
\item Text to SQL query is a popular task
  \begin{itemize}
  \item \eng{Which is the longest verb?}
  \item[\ensuremath{\rightarrow}]
\begin{verbatim}
    SELECT word, pos FROM word
    WHERE POS GLOB 'V*' AND
    LENGTH(word) =
    (SELECT MAX(LENGTH(word)) FROM word
    WHERE POS GLOB 'V*');
\end{verbatim}
    
    
  \end{itemize}

\end{itemize}

% \myslide{Acknowledgments}


% \begin{itemize}
%   \item Thanks to Stefan Th. Gries (University of California, Santa
% \end{itemize}



\end{document}


%%% Local Variables: 
%%% coding: utf-8
%%% mode: latex
%%% TeX-PDF-mode: t
%%% TeX-engine: xetex
%%% LaTeX-section-list:  (("myslide" 1))
%%% End: 