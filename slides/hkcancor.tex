\PassOptionsToPackage{xetex}{xcolor}
\PassOptionsToPackage{xetex}{graphicx}
\documentclass[a4paper,landscape,headrule,footrule,xetex]{foils}


%%
%%% macros for 2009 Semester 1 HG 803
%%%
\newcommand{\logo}{~}
\newcommand{\header}[3]{%
  \title{\vspace*{-2ex} \large HG3051  Corpus Linquistics
    \\[2ex] \Large  \emp{#2} \\ \emp{#3}}
  \author{\blu{Francis Bond}   \\ 
    \normalsize  \textbf{Division of Linguistics and Multilingual Studies}\\
    \normalsize  \url{http://www3.ntu.edu.sg/home/fcbond/}\\
    \normalsize  \texttt{bond@ieee.org}}
  \MyLogo{HG3051 (2018)}
  \renewcommand{\logo}{#2}
  \hypersetup{
    pdfinfo={
      Author={Francis Bond},
      Title={#1: #2},
      Subject={HG3051: Corpus Linguistics},
      Keywords={Corpus Linguistics},
      License={CC BY 4.0}
    }
  }
  \date{#1 \\ \url{https://github.com/bond-lab/Corpus-Linguistics}}
}

\usepackage{fontenc}
\usepackage{polyglossia}
\setmainlanguage{english}
\setmainfont{TeX Gyre Pagella}
%\setmainfont{Linux Libertine}
%\setmainfont{Charis SIL}
\newfontfamily{\ipafont}{Gentium}
\newcommand{\ipa}[1]{{\ipafont\selectfont #1}}
\usepackage{xeCJK}

\setCJKmainfont{Noto Sans CJK SC}
\setCJKsansfont{Noto Sans CJK SC}



\usepackage{xcolor}
\usepackage{graphicx}
\newcommand{\blu}[1]{\textcolor{blue}{#1}}
\newcommand{\grn}[1]{\textcolor{green}{#1}}
\newcommand{\hide}[1]{\textcolor{white}{#1}}
\newcommand{\emp}[1]{\textcolor{red}{#1}}
\newcommand{\txx}[1]{\textbf{\textcolor{blue}{#1}}}
\newcommand{\lex}[1]{\textbf{\mtcitestyle{#1}}}

\usepackage{pifont}
\renewcommand{\labelitemi}{\textcolor{violet}{\ding{227}}}
\renewcommand{\labelitemii}{\textcolor{purple}{\ding{226}}}

\newcommand{\subhead}[1]{\noindent\textbf{#1}\\[5mm]}

\newcommand{\Bad}{\emp{\raisebox{0.15ex}{\ensuremath{\mathbf{\otimes}}}}}
\newcommand{\bad}{*}

\newcommand{\com}[1]{\hfill \textnormal{(\emp{#1})}}%
\newcommand{\cxm}[1]{\hfill \textnormal{(\txx{#1})}}%
\newcommand{\cmm}[1]{\hfill \textnormal{(#1)}}%

\usepackage{relsize,xspace}
\newcommand{\into}{\ensuremath{\rightarrow}\xspace}
\newcommand{\ent}{\ensuremath{\Rightarrow}\xspace}
\newcommand{\nent}{\ensuremath{\not\Rightarrow}\xspace}
\newcommand{\tot}{\ensuremath{\leftrightarrow}\xspace}
\usepackage{url}
\newcommand{\lurl}[1]{\MyLogo{\url{#1}}}

\usepackage{mygb4e}
\let\eachwordone=\itshape
\newcommand{\lx}[1]{\textbf{\textit{#1}}}

%\usepackage{times}
%\usepackage{nttfoilhead}
\newcommand{\myslide}[1]{\foilhead[-25mm]{\raisebox{12mm}[0mm]{\emp{#1}}}\MyLogo{\logo}}
\newcommand{\myslider}[1]{\rotatefoilhead[-25mm]{\raisebox{12mm}[0mm]{\emp{#1}}}}
%\newcommand{\myslider}[1]{\rotatefoilhead{\raisebox{-8mm}{\emp{#1}}}}

\newcommand{\section}[1]{\myslide{}{\begin{center}\Huge \emp{#1}\end{center}}}



\usepackage[lyons,j,e,k]{mtg2e}
\renewcommand{\mtcitestyle}[1]{\textcolor{teal}{\textsl{#1}}}
%\renewcommand{\mtcitestyle}[1]{\textsl{#1}}
\newcommand{\chn}{\mtciteform}
\newcommand{\cmn}{\mtciteform}
\newcommand{\iz}[1]{\textup{\texttt{\textcolor{blue}{\textbf{#1}}}}}
\newcommand{\rel}[1]{\textsc{\color{blue}{#1}}}
\newcommand{\wn}[3]{\lex{#1}\ensuremath{_{#2:#3}}}
\newcommand{\con}[1]{\textsc{#1}}
\newcommand{\gm}{\textsc}
\usepackage[normalem]{ulem}
\newcommand{\ul}{\uline}
\newcommand{\ull}{\uuline}
\newcommand{\wl}{\uwave}
\newcommand{\vs}{\ensuremath{\Leftrightarrow}~}
\usepackage[hidelinks]{hyperref}
\hypersetup{
     colorlinks,
     linkcolor={blue!50!black},
     citecolor={red!50!black},
     urlcolor={blue!80!black}
}
%%%
%%% Bibliography
%%%
\usepackage{natbib}
%\usepackage{url}
\usepackage{bibentry}
%%% From Tim
\newcommand{\WMngram}[1][]{$n$-gram#1\xspace}
\newcommand{\infers}{$\rightarrow$\xspace}

\usepackage{bibentry}
\renewcommand{\cite}{\bibentry}

%\header{Lecture 4}{Survey of Available Corpora}{}

\usepackage{pst-node}
\newcommand{\sa}[2]{\rnode{c#1}{\iz{#2}}}%\nodebox{c#1}}

%\usepackage{hieroglf}
\usepackage{wasysym}
%\newcommand{\grn}[1]{\textcolor{PineGreen}{#1}}
\newcommand{\ont}[1]{\textcolor{blue}{#1}}
\newcommand{\jcy}[1]{\textcolor{orange}{#1}}
\newcommand{\lxd}[1]{\textcolor{brown}{#1}}

\newcommand{\hinoki}{\grn{Hinoki}\xspace}
\newcommand{\lexeed}{\lxd{Lexeed}\xspace}
\newcommand{\jacy}{\jcy{JACY}\xspace}
\newcommand{\onto}{\ont{Ontology}\xspace}
%\newcommand{\itsdb}{\textsf{[incr tsdb()]}\xspace}
\newcommand{\GT}{Goi-Taikei\xspace}


\begin{document}
\bibliographystyle{apalike}
\nobibliography{abb,mtg,nlp,ling}

\myslide{Corpus of Hong Kong Cantonese 香港粵語語料庫}
\MyLogo{Francis Bond \url{<bond@ieee.org>} HG3051 lab2}
\begin{itemize}
\item 180,000-word corpus of Cantonese Speech
\item 52 spontaneous conversations
\item 42 radio programmes
\item Transcribed (UTF-8); Transliterated Segmented; POS tagged
\item English translation described in paper, not in downloadable corpus
\item Available directly for download (no explicit license)
\\ \url{http://compling.hss.ntu.edu.sg/hkcancor/}
\item Produced by Luke Kang Kwong and ML Wong
\end{itemize}
\myslide{Creation}
\begin{itemize}
\item 30 hours of recordings (March 1997 --- August 1998)
\item Native speakers of Cantonese
\item ordinary settings with family members, friends and colleagues talking with each other freely on everyday topics such as current affairs, work and study, and personal hobbies
\item Some parts selected
\end{itemize}

\myslide{Meta-Data/Annotation}
\begin{itemize}
\item Meta-Data
  \begin{itemize}
  \item Tape number (of recording);  Date of recording
  \item Number of Speakers; List of Speakers (Code-Sex-Age-Origin) \\
(e.g. A-M-22-HK says A is a 22-year-old male speaker from Hong Kong)
  \end{itemize}
\item Annotation
  \begin{itemize}
  \item Each Utterance has the speaker code
  \item Utterances are segmented, POS tagged and transliterated
    \begin{verbatim}
基本上/d/ge3i1bun2soeng6/
哩個/r/ni1go3/ze
...
\end{verbatim}
  \end{itemize}
\item The whole corpus is wrapped in xml (but not very well)
\end{itemize}


\myslide{Usage}
\begin{itemize}
\item Used to examine the uses of the frequently used sentence final particles w\v{o} and b\v{o} in the 1990s in Hong Kong Cantonese by examining speech data.
\item Question: are w\v{o} (喎) and b\v{o} (噃) variant forms?
\item Answer: No
  \begin{quotation}
    ``{[\ldots]} the two SFPs carry and serve different meanings and functions in modern Hong Kong Cantonese, and thus they are not exactly the same particles and not interchangeable as previously assumed.''
    \citep[p21]{Leung:2010}
  \end{quotation}
\item Also used as a corpus in the \textit{PyCantonese} Project:
  Working with Cantonese corpus data using Python, by Jackson L. Lee (\url{https://github.com/pycantonese/pycantonese})

\end{itemize}

\myslide{References}
\begin{itemize}
\item \cite{Luke:Wong:2015}
\item \cite{Wong:2006}
\item \cite{Leung:2010}
\end{itemize}

\myslide{ISLRN}
\noindent{\footnotesize \begin{tabular}{lp{0.8\textwidth}}
Title &  HKcancor \\
Full Title &  Hong Kong Cantonese Corpus \\
Resource Type &  Speech \\
Source/URL &  \url{compling.hss.ntu.edu.sg/hkcancor} \\
Format/MIME Type &  text/xml \\
Size/Duration &   230,000 words \\
Access Medium &  online \\
Description &  The Hong Kong Cantonese Corpus was collected from
             transcribed conversations that were recorded between
             March 1997 and August 1998. About 230,000 Chinese
             words were collected in the annotated corpus. It
             contains recordings of spontaneous speech (51 texts)
             and radio programmes (42 texts), which involve 2 to
             4 speakers, with 1 text of monologue. The text were
             word-segmented, annotated with part-of-speech
             tagging and Cantonese pronunciation using the 
             romanisation scheme of the Linguistic Society of Hong
             Kong (LSHK). \\

Version &  ?  \\
Media Type &  Transcribed Speech \\
Language &  yue (Cantonese) \\
Resource Creator &  KK Luke \\
Distributor &  KK Luke \\
Rights Holder &  KK Luke
\end{tabular}}

\end{document}


%%% Local Variables: 
%%% coding: utf-8
%%% mode: latex
%%% TeX-PDF-mode: t
%%% TeX-engine: xetex
%%% LaTeX-section-list:  (("myslide" 1))
%%% End: 
