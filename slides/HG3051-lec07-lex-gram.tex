\documentclass[a4paper,landscape,headrule,footrule,dvips]{foils}

%%
%%% macros for 2009 Semester 1 HG 803
%%%
\newcommand{\logo}{~}
\newcommand{\header}[3]{%
  \title{\vspace*{-2ex} \large HG3051  Corpus Linquistics
    \\[2ex] \Large  \emp{#2} \\ \emp{#3}}
  \author{\blu{Francis Bond}   \\ 
    \normalsize  \textbf{Division of Linguistics and Multilingual Studies}\\
    \normalsize  \url{http://www3.ntu.edu.sg/home/fcbond/}\\
    \normalsize  \texttt{bond@ieee.org}}
  \MyLogo{HG3051 (2018)}
  \renewcommand{\logo}{#2}
  \hypersetup{
    pdfinfo={
      Author={Francis Bond},
      Title={#1: #2},
      Subject={HG3051: Corpus Linguistics},
      Keywords={Corpus Linguistics},
      License={CC BY 4.0}
    }
  }
  \date{#1 \\ \url{https://github.com/bond-lab/Corpus-Linguistics}}
}

\usepackage{fontenc}
\usepackage{polyglossia}
\setmainlanguage{english}
\setmainfont{TeX Gyre Pagella}
%\setmainfont{Linux Libertine}
%\setmainfont{Charis SIL}
\newfontfamily{\ipafont}{Gentium}
\newcommand{\ipa}[1]{{\ipafont\selectfont #1}}
\usepackage{xeCJK}

\setCJKmainfont{Noto Sans CJK SC}
\setCJKsansfont{Noto Sans CJK SC}



\usepackage{xcolor}
\usepackage{graphicx}
\newcommand{\blu}[1]{\textcolor{blue}{#1}}
\newcommand{\grn}[1]{\textcolor{green}{#1}}
\newcommand{\hide}[1]{\textcolor{white}{#1}}
\newcommand{\emp}[1]{\textcolor{red}{#1}}
\newcommand{\txx}[1]{\textbf{\textcolor{blue}{#1}}}
\newcommand{\lex}[1]{\textbf{\mtcitestyle{#1}}}

\usepackage{pifont}
\renewcommand{\labelitemi}{\textcolor{violet}{\ding{227}}}
\renewcommand{\labelitemii}{\textcolor{purple}{\ding{226}}}

\newcommand{\subhead}[1]{\noindent\textbf{#1}\\[5mm]}

\newcommand{\Bad}{\emp{\raisebox{0.15ex}{\ensuremath{\mathbf{\otimes}}}}}
\newcommand{\bad}{*}

\newcommand{\com}[1]{\hfill \textnormal{(\emp{#1})}}%
\newcommand{\cxm}[1]{\hfill \textnormal{(\txx{#1})}}%
\newcommand{\cmm}[1]{\hfill \textnormal{(#1)}}%

\usepackage{relsize,xspace}
\newcommand{\into}{\ensuremath{\rightarrow}\xspace}
\newcommand{\ent}{\ensuremath{\Rightarrow}\xspace}
\newcommand{\nent}{\ensuremath{\not\Rightarrow}\xspace}
\newcommand{\tot}{\ensuremath{\leftrightarrow}\xspace}
\usepackage{url}
\newcommand{\lurl}[1]{\MyLogo{\url{#1}}}

\usepackage{mygb4e}
\let\eachwordone=\itshape
\newcommand{\lx}[1]{\textbf{\textit{#1}}}

%\usepackage{times}
%\usepackage{nttfoilhead}
\newcommand{\myslide}[1]{\foilhead[-25mm]{\raisebox{12mm}[0mm]{\emp{#1}}}\MyLogo{\logo}}
\newcommand{\myslider}[1]{\rotatefoilhead[-25mm]{\raisebox{12mm}[0mm]{\emp{#1}}}}
%\newcommand{\myslider}[1]{\rotatefoilhead{\raisebox{-8mm}{\emp{#1}}}}

\newcommand{\section}[1]{\myslide{}{\begin{center}\Huge \emp{#1}\end{center}}}



\usepackage[lyons,j,e,k]{mtg2e}
\renewcommand{\mtcitestyle}[1]{\textcolor{teal}{\textsl{#1}}}
%\renewcommand{\mtcitestyle}[1]{\textsl{#1}}
\newcommand{\chn}{\mtciteform}
\newcommand{\cmn}{\mtciteform}
\newcommand{\iz}[1]{\textup{\texttt{\textcolor{blue}{\textbf{#1}}}}}
\newcommand{\rel}[1]{\textsc{\color{blue}{#1}}}
\newcommand{\wn}[3]{\lex{#1}\ensuremath{_{#2:#3}}}
\newcommand{\con}[1]{\textsc{#1}}
\newcommand{\gm}{\textsc}
\usepackage[normalem]{ulem}
\newcommand{\ul}{\uline}
\newcommand{\ull}{\uuline}
\newcommand{\wl}{\uwave}
\newcommand{\vs}{\ensuremath{\Leftrightarrow}~}
\usepackage[hidelinks]{hyperref}
\hypersetup{
     colorlinks,
     linkcolor={blue!50!black},
     citecolor={red!50!black},
     urlcolor={blue!80!black}
}
%%%
%%% Bibliography
%%%
\usepackage{natbib}
%\usepackage{url}
\usepackage{bibentry}
%%% From Tim
\newcommand{\WMngram}[1][]{$n$-gram#1\xspace}
\newcommand{\infers}{$\rightarrow$\xspace}

\usepackage{bibentry}
\renewcommand{\cite}{\bibentry}

\header{Lecture 7}{Lexical and Grammatical Studies, Variation}{}

\usepackage{pst-node}
\newcommand{\sa}[2]{\rnode{c#1}{\iz{#2}}}%\nodebox{c#1}}

%\usepackage{hieroglf}
\usepackage{wasysym}
%\newcommand{\grn}[1]{\textcolor{PineGreen}{#1}}
\newcommand{\ont}[1]{\textcolor{blue}{#1}}
\newcommand{\jcy}[1]{\textcolor{orange}{#1}}
\newcommand{\lxd}[1]{\textcolor{brown}{#1}}

\newcommand{\hinoki}{\grn{Hinoki}\xspace}
\newcommand{\lexeed}{\lxd{Lexeed}\xspace}
\newcommand{\jacy}{\jcy{JACY}\xspace}
\newcommand{\onto}{\ont{Ontology}\xspace}
%\newcommand{\itsdb}{\textsf{[incr tsdb()]}\xspace}
\newcommand{\GT}{Goi-Taikei\xspace}


\begin{document}
\begin{CJK}{UTF8}{min}
\bibliographystyle{apalike}
\nobibliography{abb,mtg,nlp,ling}
\maketitle


\myslide{Overview}

\begin{itemize} 
\item Revision  of DIY Corpora
  \begin{itemize} 
  \item DIY Corpora
  \item Corpus Tools
  \item Processing Raw Text
  \item SQL
  \end{itemize}
\item Lexical Studies
\item Grammatical Studies
\item Variation
\end{itemize}
%%% 
%%% this changes each year, so keep separate
%%%
\include{schedule}

%%%
\myslide{Revision  of DIY Corpora}
%%%


\myslide{DIY Corpora}

\begin{enumerate}
\item Decide what you want to study
\\ see if you can do it with existing resources
\\ No \frownie
\item Collect data that fits your needs
  \begin{itemize}
  \item Speech (expensive)
  \item Text (easy to get, hard to do legally)
  \end{itemize}
\item Process it
  \begin{itemize}
  \item Clean up
  \item Mark up
  \item Annotation
  \end{itemize}
\end{enumerate}

\myslide{Web as Corpus}
\myslide{Two Approaches to using the Web as a Corpus}
\begin{itemize}
\item \blu{Direct Query}: Search Engine as Query tool and WWW as corpus?
\\  (Objection: Results are not reliable)
\begin{itemize}
\item Population and exact hit counts are unknown → no statistics
possible.
\item Indexing does not allow to draw conclusions on the data.
\item[\Bad] Google is missing functionalities that linguists /
lexicographers would like to have.
\end{itemize}
\item \blu{Web Sample}: Use search engine to download data from the
net and build a corpus from it.
\begin{itemize}
\item known size and exact hit counts → statistics possible.
\item people can draw conclusions over the included text types.
\item (limited) control over the content.
\item[\Bad] sparser data
\end{itemize}
\end{itemize}


\myslide{Structured Query Language}

A special-purpose programming language designed for managing data held
in a relational database management system (RDBMS)

\begin{itemize} \addtolength{\itemsep}{-1ex}
\item The simplest query
  \begin{itemize}
  \item \txx{SELECT} desired attributes
  \item \txx{FROM} one or more tables
  \item \txx{WHERE} condition applies about the records in the tables
  \end{itemize}
\item What lemmas are there associated with the word \eng{does}?

  \begin{tabular}{l|ll}
\textbf{QUERY}  &   \textbf{word} & \textbf{lemma} \\ \hline
SELECT word, lemma  &   does & do \\
FROM word           &   does & do \\
WHERE word = 'does' &   does & doe \\
    & \ldots
  \end{tabular}
\end{itemize}

\myslide{Queries can be rather complicated}

\begin{itemize}
\item Tell me more about frequent common nouns
\begin{verbatim}
SELECT count(word), count(DISTINCT word),
       MIN(LENGTH(word)), 
       MAX(LENGTH(word)), AVG(LENGTH(word))
FROM word
WHERE word 
IN (SELECT word
    FROM word
    WHERE POS in ('NN', 'NNS') 
    GROUP BY word
    ORDER BY COUNT(word) DESC
    LIMIT 10)

\end{verbatim}
  \begin{tabular}{lllll}
    \textbf{count}  & \textbf{distinct} &\textbf{min}  &  \textbf{max} & \textbf{avg} \\ \hline
    1096 & 10 & 3 & 10 & 5.13
  \end{tabular}
\item First find the ten most common words, then do things to them
  \begin{itemize}
  \item \emp{the trick is to write simple queries first, and then combine them}
  \end{itemize}
\item Note that more common words are shorter (as we would expect)
\end{itemize}
\begin{center}
  SQL allows you to search for a very wide variety of things
\end{center}


%%%
%%%
%%%
\section{Corpus Studies of Lexicography}

\myslide{\lex{big}, \lex{large}, \lex{great}}
\MyLogo{\citet[\S~2.6]{Biber:Conrad:Reppen:1998}}

\begin{itemize}
\item Same syntax (all adjectives)
\item Similar meaning:
  \begin{itemize}
  \item \lex{large, big} --- ``above average in size or number or quantity or magnitude or extent'' \eng{a large city; set out for the big city; a large sum; a big (or large) barn;  a large family;  big businesses;  a big expenditure; a large number of newspapers; a big group of scientists; large areas of the world}
  \item \lex{great} --- ``relatively large in size or number or extent; larger than others of its kind'' \eng{a great juicy steak; a great multitude; the great auk; a great old oak; a great ocean liner; a great delay} 
\end{itemize}
\item How do they differ?
\end{itemize}

\myslide{Distribution of \lex{big}, \lex{large}, \lex{great}}
\MyLogo{(After table 2.5 \citep[p 60]{Biber:Conrad:Reppen:1998})}

\begin{tabular}{lrrr}
 &   Academic &   Fiction & Combined\\
\hline
\lex{big}    &  31 & 408 & 230 \\
\lex{large}  & 605 & 232 & 408 \\
\lex{great}  & 284 & 490 & 393 
\end{tabular}
\\ (frequency/million words)

\begin{itemize}
\item Counts from Longman-Lancaster Corpus
  \begin{itemize}
  \item Academic Text: 2.7 Million Words
  \item Fiction: 3.0 Million Words
  \end{itemize}
\end{itemize}

\myslide{Immediate Right Collocates}
\MyLogo{(frequency/million words)}
\begin{center}
  \begin{tabular}{lrlrlr}
& &   \multicolumn{2}{c}{\emp{Academic}}\\[1ex]
    \multicolumn{2}{c}{\lex{big}} &   
    \multicolumn{2}{c}{\lex{large}} &
    \multicolumn{2}{c}{\lex{great}}\\
\hline
    enough & 2.2 & number & 48.3 & deal & 44.6 \\
    traders & 1.1 & numbers & 31.3 & importance & 12.5 \\
    &     & scale & 18.0 & number &  8.9 \\
    &     & and & 28.0 & majority & 8.1 \\ 
    &     & enough & 15.9 & variety & 7.0 \\ [1ex]

 & &   \multicolumn{2}{c}{\emp{Fiction}}\\[1ex]
    \multicolumn{2}{c}{\lex{big}} &   
    \multicolumn{2}{c}{\lex{large}} &
    \multicolumn{2}{c}{\lex{great}}\\
\hline
    man  & 9.6 & and & 15.2 & deal & 40.4 \\
    enough & 8.9 & black & 4.3 & man & 6.6 \\
    and & 8.3 & enough & 3.6 & burrow &  5.6 \\
    house    & 7.6 & room & 2.7 & big & 4.6 \\ 
    big        & 7.0 & white & 2.7 & aunt & 4.3 \\ 
  \end{tabular}
\end{center}

\myslide{Discussion \lex{big}, \lex{large}, \lex{great}}

\begin{itemize}
\item \lex{big}  mainly for concrete things
\item \lex{large}  mainly for amounts and numbers
\item \lex{great} similar to \lex{large} but many special senses
  \begin{itemize}
  \item \lex{great deal}  
  \item \lex{great man}  
  \item \lex{great burrow} (\textit{Watership Down})
  \item \lex{great} \iz{relative}
  \end{itemize}
also use as intensifier \eng{great big, great importance}
\end{itemize}

The dictionary definition does not really tell us this.

\section{Corpus Studies of Morphology}

\myslide{Distribution and Function of Nominalizations}
\MyLogo{\citet[\S~3.2]{Biber:Conrad:Reppen:1998}}

\begin{itemize}
\item Investigate how common normalizations are in different registers
\item Count four common derivational suffixes:  \lex{-[ts]ion, -ment, -ness, -ity}
\item In three registers: Academic, Fiction, Speech
\item Search for words ending in \eng{tion, sion, ity, \ldots}
\\ with a stop list: \lex{nation, station, city, \ldots}
\item First run the matcher, then add stop words, then rerun, \ldots
\\ hard to do with a web interface
\end{itemize}

\myslide{Results (1)}
\MyLogo{(After table 3.1 \citep[p 60]{Biber:Conrad:Reppen:1998})}
\noindent \textbf{Nominalizations per thousand words across registers}
\\ \begin{tabular}{lrrr}
 Academic & Fiction & Speech \\ \hline
 44.0   & 11.2    &  11.3   
\end{tabular}


\begin{itemize}
\item Nominalizations much more common in Academic text
\item A few words very common (more than 500 per million)
\\ \eng{movement, activity, information,$^*$ development, relation, equation}
\begin{itemize}
\item If \lex{movement} has occurred recently \ldots \com{Academic}
\item Garth [\ldots] \lex{moved} his hand crabwise along the table.
\com{Fiction}
\item When we \lex{moved} into the new house  \ldots \com{Speech}
\end{itemize}
\item Academic text focuses on generalized processes
\item Speech and fiction focus on a specific person doing some activity
\end{itemize}


\myslide{Results (2)}
\MyLogo{(After table 3.2 \citep[p 63]{Biber:Conrad:Reppen:1998})}
\noindent \textbf{Proportions of different suffixes across registers}
\\[2ex] \begin{tabular}{lrrr}
  suffix   & Academic & Fiction & Speech \\\hline
  \lex{-[ts]ion} & 68\%     & 51\%    &  56\% \\  
  \lex{-ment}   & 15\%     & 21\%    &  24\% \\  
  \lex{-ness}    &  2\%     & 13\%    &   5\% \\  
  \lex{-ity}     & 15\%     & 15\%    &  15\% \\  
\end{tabular}

\begin{itemize}
\item  \lex{-[ts]ion} more common in Academic (but common everywhere)
\item \lex{-ment} commoner in Fiction and Speech
\item \lex{-ness} common in Fiction
\end{itemize}
\myslide{Discussion}

\begin{itemize}
\item  \lex{-[ts]ion} more common in Academic (but common everywhere)
  \\ basic use is to make an action non-agentive
  \begin{itemize}
  \item \eng{It provides a direct \ul{indication} of fuel \ul{consumption}.}
  \end{itemize}
\item \lex{-ment} often used for mental states
  \\ \eng{agreement, amazement, embarrassment} \com{Fiction}
  \begin{itemize}
  \item \eng{Patrick shrugged in \ul{embarrassment}.} 
  \end{itemize}
\item \lex{-ness} used for personal qualities
  \\ \eng{bitterness, happiness, politeness} \com{Fiction}
  \begin{itemize}
  \item \eng{The \ul{bitterness} in his heart was mixed with \ldots . }
  \end{itemize}
\end{itemize}

It would be good if we could automatically divide the words according
to their semantic field (which we can approximate with WordNet, \ldots)
%%FIXME expand

\section{Corpus Studies of Syntax}
\MyLogo{}
\myslide{\lex{begin} vs \lex{start}}
\MyLogo{\citet[\S~4.3]{Biber:Conrad:Reppen:1998}}

\begin{itemize}
\item \lex{begin} and \lex{start} are very similar in meaning
  \begin{itemize}
  \item \lex{get down, begin, get, start out, start, set about, set
      out, commence} --- ``take the first step or steps in carrying
    out an action'' \eng{We began working at dawn; Who will start?;
      Get working as soon as the sun rises!; The first tourists began
      to arrive in Cambodia; He began early in the day; Let's get down
      to work now}
  \item \lex{begin, start} 
 --- ``have a beginning, in a temporal, spatial, or evaluative sense''
 \eng{The DMZ begins right over the hill; The second movement begins
   after the Allegro; Prices for these homes start at \$250,000}
  \item \lex{begin, lead off, start, commence} --- ``set in motion, cause to start'' 
    \eng{the U.S. started a war in the Middle East; the Iraqis began
      hostilities; begin a new chapter in your life}
  \end{itemize}
\end{itemize}


\myslide{\lex{begin} vs \lex{start}}
\MyLogo{}

\begin{itemize}
\item \lex{begin} and \lex{start} are very similar in possible usage
  \begin{itemize}
  \item \txx{intransitive}
    \\ \eng{I had better rest before we \ul{begin/start}}
  \item \txx{transitive (NP)}
    \\ \eng{I will  \ul{begin/start} the lecture at 18:00}
  \item \txx{transitive (VP:ing)}
    \\ \eng{I will \ul{begin/start} lecturing at 18:00}
  \item \txx{transitive (VP:to)}
    \\ \eng{I will \ul{begin/start} to lecture at 18:00}
  \end{itemize}
\item So how do they differ?
\end{itemize}

\myslide{\lex{begin} vs \lex{start}}
\MyLogo{}
\begin{itemize}
\item Automatically tag text from two registers 
(Longman-Lancaster Fiction and Academic)
\\ \texttt{V (ADV)? NP} $\Rightarrow$ T (transitive) 
\\ \texttt{V (ADV)? to} $\Rightarrow$ TCLS (to clause transitive)
\\ \texttt{V (ADV)? V+ing } $\Rightarrow$ ING (-ing clause transitive)
\\ \textbf{else}: $\Rightarrow$ I (intransitive)
\item Aim for 250 samples, take every third 
\item Hand correct the initial sample
\end{itemize}

\myslide{Example}

\begin{verbatim}
00018.FCT
<valency=TCLS (I)
hath her in thrall. "After a minute, the trio
==> began
rather carefully to cross the room


00021.FCT
<valency=ING (I)
station, shops, roadhouses, all closed. A dog
==> began
barking and , having begun , went on.
\end{verbatim}

\myslide{Corrected Results}

\begin{tabular}{lrrrrr}
   & Intransitive & +NP & +to & +ing  \\
\hline
\lex{begin} \\
Fiction & 22\% & 3\% & 72\% & 4\% \\ 
Academic & 43\% & 12\% & 34\% &12\% \\ 
\lex{start} \\
Fiction & 40\% & 22\% & 20\% & 18\% \\ 
Academic & 64\% & 16\% & 15\% & 6\% \\ 
\end{tabular}

\begin{itemize}
\item \lex{start} is more common as intransitive
\item \lex{begin} is more common as \lex{to}-transitive
\end{itemize}

(After table 4.3 \citep[p 98]{Biber:Conrad:Reppen:1998})
\myslide{Discussion}

Typically \lex{start} is used to show the onset of a process, often
with an adverb
\begin{itemize}
\item \eng{The soil formation  process may \ul{start} again in the fresh material}
\item \eng{The train \ul{started} down the hill}
\end{itemize}

\lex{begin} is used with more concrete agents
\begin{itemize}
\item \eng{Then I \ul{began}  to laugh a bit.}
\item \eng{The original mass of gas cooled and \ul{began} to contract.}
\end{itemize}

Because the corpus doesn't mark \txx{animacy} or \txx{concrete agent} these
statements are weak: we can't really make predictions or measure correlation.

\myslide{Can we do better?}
\lurl{http://nltk.ldc.upenn.edu:8080/ts/search}
\begin{itemize}
\item Treebanks exist for some languages
\item We can search some English treebanks
\\ \lurl{http://nltk.ldc.upenn.edu:8080/ts/search}
\\ \verb|//VP/VB/begin[->S/VP/TO/to]|
\\ \verb|//VP/VB/start[->S/VP/VBG]|
\item This can also be done offline to get counts
\end{itemize}

\myslide{What about SQL?}

We can look at a word and the next word by \txx{joining} a table to itself

\begin{itemize}
\item Transitive (V N) 
\begin{verbatim}
SELECT a.word, b.word, b.pos
FROM word AS a JOIN word AS b 
  ON a.sid=b.sid AND a.wid=b.wid-1
WHERE a.lemma='start' AND b.pos GLOB 'N*'
\end{verbatim}
\item Transitive (VP:ing)
\begin{verbatim}
SELECT a.word, b.word, b.pos
FROM word AS a JOIN word AS b 
  ON a.sid=b.sid AND a.wid=b.wid-1
WHERE a.lemma='start' AND b.pos='VBG'
\end{verbatim}
\newpage
\item Transitive (VP:to)
\begin{verbatim}
SELECT a.word, b.word, b.pos
FROM word AS a JOIN word AS b 
  ON a.sid=b.sid AND a.wid=b.wid-1
WHERE a.lemma='start' AND b.pos='TO'
\end{verbatim}
\item Intransitive (remainder)
\end{itemize}

Regular expressions are better for this, SQL is not very good at one
or none.  But it is easy to write a few queries and add them together.
\newpage
\begin{itemize}
\item Transitive (V ADV N) (none in eng.db)
\begin{verbatim}
SELECT a.word, b.word, b.pos
FROM word AS a JOIN word AS b JOIN word as c
  ON a.sid=b.sid AND a.wid=b.wid-1 AND a.wid=c.wid-2
WHERE a.lemma='start' AND b.pos GLOB 'R*' AND c.pos GLOB 'N*'
\end{verbatim}
\end{itemize}

Try to do these as nested loops!


\myslide{JOINS}
\MyLogo{Can be very, very slow}

An SQL \txx{JOIN} clause is used to combine rows from two or more
tables, based on common fields between them.

\begin{itemize}\addtolength{\itemsep}{-1ex}
\item (INNER) JOIN: Returns all rows with a  match in BOTH tables
\item LEFT JOIN: Return all rows from the left table, and matched rows from the right table
\item RIGHT JOIN: Return all rows from the right table, and matched rows from the left table
\item FULL JOIN: Return all rows with a match in EITHER table
\end{itemize}
\begin{verbatim}
SELECT column_name(s)
FROM table1
JOIN table2
ON table1.column_name=table2.column_name;
\end{verbatim}


\myslide{\lex{little} vs \lex{small}} 
\MyLogo{\citet[\S~4.2]{Biber:Conrad:Reppen:1998}}
\begin{itemize}
\item \lex{little} and \lex{small} are nearly synonymous
\item WordNet 3.0 has them share 4 synsets out of 10 for \lex{small} and 8 for \lex{little}
  \begin{itemize}
  \item \lex{small, little} --- ``limited or below average in number or quantity or magnitude or extent'' \eng{a little dining room; a little house; a small car; a little (or small) group}
  \item \lex{little, small} --- ``(of children and animals) young, immature'' \eng{what a big little boy you are; small children}
  \item \lex{little, minuscule, small} --- ``lowercase'' \eng{little a; small a; e.e.cummings's poetry is written all in minuscule letters}
  \item \lex{little, small} --- ``(of a voice) faint'' \eng{a little voice; a still small voice}
\end{itemize}
\item Yet they differ semantically and syntactically
\end{itemize}

\myslide{Syntax: predicative vs attributive}
\MyLogo{}
\begin{itemize}
\item \txx{Predicative}
  \\ \eng{When I was \ul{little/small}, I couldn't say ``hospital''}
\item \txx{Attributive}
  \\ \eng{It's only a \ul{little/small} puppy}
\item Are they used in the same way?
  \begin{itemize}
  \item 5 million words of conversation from BNC
  \item 5 million words of academic text from Longman-Lancaster
  \end{itemize}
\end{itemize}

\myslide{How to find usage examples?}

\begin{itemize}
\item Automatic pass (collect data matching patterns)
\item Hand checking of a sample
\item Re-weigh counts
\end{itemize}

\myslide{Automatic pass}
\begin{itemize}
\item Match patterns against the corpus
\begin{itemize}
\item \txx{Predicative}
  \\ \eng{When I was \{little/small\}, I couldn't say ``hospital''}
  \\ \texttt{be (ADV)? (little$\|$small)}
\item \txx{Attributive}
  \\ \eng{It's only a \ul{little/small} puppy}
 \\  \texttt{(little$\|$small) (ADJ)? NN}
\item \txx{No tag} (the remainder)
\end{itemize}
\item Store the results
\begin{verbatim}
Type = Atrb; File = 00116.TEC
section at the center of each lesion is a 
-----> small
bronchus containing lungworms and ...
\end{verbatim}
\end{itemize}

\myslide{Initial Results}

\begin{tabular}{llrrrr}
  Type & Word & Atrb & Pred & No Tag & Total \\ 
\hline\hline
Conversation & little & 2,101 & 104 & 405 & 2,610 \\
             & small &   399 & 72 & 158 & 629 \\   
\hline
Academic & little & 1,033 & 65 & 411 & 1,509 \\
             & small & 2,557 & 316 & 399 & 3,272 \\
\hline
Total  & & 6,090& 557 & 1,373 & 8,020 
\end{tabular}

(After table 4.1 \citep[p 91]{Biber:Conrad:Reppen:1998})

\begin{itemize}
\item More \txx{no tag} than \txx{predicative}
\item So we can't be confident
\item Look at a sample (about a hundred) of each group
  
\end{itemize}

\myslide{Some example errors}
\MyLogo{Simplified}
\begin{verbatim}
Type = No tag; File = 00116.TEC  -> attributive
Shut up you 
-----> little
... cow

Type = No tag; File = 00117.TEC  -> predicative
It is by no means
-----> small
for a brachiapod

Type = No tag; File = 02316.TEC  -> attributive
A: Cause they have 
-----> little
B: We
A: milk bottles
\end{verbatim}



\myslide{Hand checking of a sample}
\MyLogo{}
\begin{enumerate}
\item Extract a random sample of occurrences
  \\ the bigger the better, make sure it is uniform
\item Analyze the grammatical feature by hand
\item Compute the proportional use of each variant in the sample
\item Multiply the total number of occurrences by these proportions
\item Adjust the original counts by the weighted counts
\end{enumerate}

\myslide{Hand checking of a sample}

\begin{itemize}
\item In this case look at a hundred from each group (6 samples)
  \begin{itemize}
  \item Consider \lex{little} in conversation
    \begin{itemize}
    \item attributive: 100\% atrb
    \item predicative:   42\% atrb; 39\% pred; 19\% other
    \item no tag: 57\% atrb; 4\% pred; 39\% other
%(/ 41 104.0) 0.3942307692307692
%(/ 44 104.0) 0.4230769230769231
    \end{itemize}
  \end{itemize}
\item Use these proportions to recalculate:
  \begin{itemize}
  \item Attributive: $2,101 + .42\times 104 + .57 \times 405 = 2,376$
  \item Predicative: $.39\times 104 + .04 \times 405 = 57$
  \end{itemize}
\end{itemize}

\myslide{Re-weighted Counts}

\begin{tabular}{llrr}
  Type & Word & Original & Weighted \\ 
  &  & \% Pred & \%  Pred \\ 
\hline\hline
Conversation & little & $5$ & $2$ \\
             & small &  $15$ & $23$ \\
\hline
Academic & little & $6$  & $<1$\\
           & small & $11$ & $13$ \\

\end{tabular}

(After table 4.2 \citep[p 93]{Biber:Conrad:Reppen:1998})

\begin{itemize}
\item Adjusted counts more accurate
\item Only had to check 600 (of 8,020)
\item But hard to reuse or go further: only a small accurate sample
  \\ parsed text is better
\end{itemize}

\myslide{Interpretation}

\begin{itemize}
\item Attributive much more common for both
  \begin{itemize}
  \item Predicative relatively more common in conversation
  \item Predicative relatively more common for \lex{small} than \lex{little}
  \end{itemize}
\item Collocation results:
  \begin{itemize}
  \item \lex{little}: concrete objects (\eng{little boy})
  \item \lex{small}: amounts  (\eng{small proportion})
  \end{itemize}
\item But predicative \lex{small} also for physical size:
  \begin{itemize}
  \item \eng{She's small and really skinny}
  \item \eng{He's really small isn't he?}
  \end{itemize}
\item We still don't really know why \frownie
 \\ corpus linguistics gives us the \emp{what}, but not the \emp{why}
\end{itemize}


\myslide{Can we do better?}
\lurl{http://nltk.ldc.upenn.edu:8080/ts/search}
\begin{itemize}
\item Treebanks exist for some languages
\item We can search some English treebanks (wikipedia)
\\ \lurl{http://nltk.ldc.upenn.edu:8080/ts/search}
\begin{itemize}
\item  \verb|//VP/ADJP/JJ/small| \com{predicative}
\item \verb|//NP/ADJP/JJ/small| + \verb|//NP/JJ/small| \com{attributive}
\end{itemize}
\item This can also be done offline to get counts
\end{itemize}




\myslide{Where do we go from here?}
\MyLogo{}
\begin{itemize}
\item Corpora show clearly that even semantically very similar words
  can show different behavior.
\item But they still don't explain why
  \begin{itemize}
  \item Hand correction limits data sizes
  \item Without semantic tags, we can't generalize automatically
  \end{itemize}
\item Corpora with more mark-up (syntax and semantics) would help
  \begin{itemize}
  \item But they are expensive, \ldots
  \end{itemize}
\end{itemize}

\myslide{Acknowledgments}
\MyLogo{HG351 (2011)}

\begin{itemize}
\item Many examples from chapters 3 and 4 of \bibentry{Biber:Conrad:Reppen:1998}
\end{itemize}
% \item Thanks to Stefan Th. Gries (University of California, Santa
%    Barbara) for his great introduction \textit{Useful statistics for
%      corpus linguistics} \url{http://www.linguistics.ucsb.edu/faculty/stgries/research/UsefulStatsForCorpLing.pdf}
%  \item Some examples taken from Ted Dunning's \textit{Surprise and
%      Coincidence - musings from the long tail}
%    \url{http://tdunning.blogspot.com/2008/03/surprise-and-coincidence.html}

%   inspiration for some of the slides (from  \textit{LING 2050 Special Topics in Linguistics: Corpus linguistics}, U Penn).
% \item Thanks to Sandra K\"{u}bler for some of the slides from her 
% \textit{RoCoLi\footnote{Romania Computational Linguistics Summer School} Course: Computational Tools for Corpus Linguistics}
% %\item Thanks to Mark Davies (BYU) for the exploration ideas.
% \item Definitions from WordNet 3.0
% \end{itemize}



%%
%% Future
%%
\clearpage
\end{CJK}
\end{document}

% Local Variables: 
% TeX-view-style: (("." "xdvi-ja %d -paper a4r -s 7"))
% LaTeX-section-list:  (("myslide" 1))
% TeX-master: t
% End:
